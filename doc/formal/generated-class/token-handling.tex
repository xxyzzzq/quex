Section \ref{sec:basics/token-queue} elaborated on the idea that the lexical
analyser communicates tokens to the user. In the {\quex} generated lexical
analyser the tokens are stored in a token-queue before they are delivered to
the caller of {\tt get\_token()}. Inside the pattern-action the {\tt send}
function group allows to send tokens:

\begin{lstlisting}
        void  &      send(const token& That);                             
        void  &      send(const token::id_type TokenID);                 
        void  &      send_n(const int N, const token::id_type TokenID);   
        void  &      send(const token::id_type TokenID, const char* Text);
        void  &      send(const token::id_type TokenID, const int Number1);
\end{lstlisting}

Figure \ref{fig:token-queue}, page \pageref{fig:token-queue}, displayed a sequence diagram depicting the
communication of tokens. Using the {\tt send} function group the writer of
pattern-action pairs does not need to care about the {\tt get\_token()}
function call. He sends tokens and leaves the responsibility of delivery
to the engine.

