Traditional lexical analyzer generators concentrate on the ASCII \cite{}
character set for encoding of character. With the raise of UTF-8 \cite{} those
generators could even be used to parse unicode character sets, since they were
unaware of escape characters.   The
problem with this approach is first of all a lack of flexibility.  UTF-8 is for
most languages, i.e. unicode code pages, a highly redundant coding. Second the
UTF-8 standard (ISO-10646) is a dynamic length coding where characters are
coded with different numbers of bytes.  Counting of columns or lexeme lengthes
is no longer trivial and is prone to consume significant computation time. 
For lex, for example, the definition of character ranges such as
$[$a-z$]$ for non-latin languages is not possible because it considers
only the two bytes around the '-' where the characters might actually be
represented by more than one byte.

\showpic
{figures/character-encoding.png}
{Character encodings involved in {\quex}'s generation process.}
{fig:character-encoding}

Under Unix, a utility called 'iconv' \cite{} is the quasi standard for
character set conversions. It is freely available for any major operating
system.  {\Quex} uses the library of this utility and has it webbed into its
buffer handling. This way {\tt quex} is able to analyze character streams of
all the hundreds of character encodings that iconv supports. It is essential to
understand that there are {\it three} different places where character
encodings are involved in the lexical analyzer generation. As shown in figure
\ref{fig:character-encodings} these are the following:

\begin{itemize}
    \item {\bf Lexical Analyzer Description:} The encoding of the lexical analyzer generator, i.e. the {\quex}-files
          which are used as input to {\quex} and {\it must} use UTF-8.
          This encoding is independent of the coding of the files the generated
          lexical analyzer takes as input.

    \item {\bf Files to Be Analyzed:} The coding
          format of the files that the generated lexical analyzer reads. Note, that
          this format is tied to a buffer. {\Quex}, though, provides the flexibility
          to switch buffers and to allow to read files in different encodings 
          with the same lexical analyzer.

    \item {\bf Internal Engine:} The internal character encoding. This is {\it always} Unicode Standard \cite{}. 
          The user, though, can specify whether one, two, or four bytes are to be used 
          for the character representation. Accordingly, it uses ASCII, UCS-2, or UCS-4
          as internal coding. Note, that this encoding determines the lexeme type. 
          It might be a good idea to adapt the character width to the definition of
          {\tt wchar\_t}\footnote{The type {\tt wchar\_t} which stands for wide character type, is usually
          32 bits, i.e. 4 bytes on Unix systems and 16 bit on Windows based platforms.}. 
        
\end{itemize}

There only three things required to get the encoding support for a particular coding. First
one needs to make sure that the {\quex} files, i.e. the files that describe the lexical
analyzer are stored in UTF-8 format by your editor. Most editors do this anyway by default.
You need to pass the option {\tt --iconv} to {\quex} when calling it. Third, the constructor
of {\quex} is to be called with the encoding name of the files you want to parse, i.e. one might
type

\begin{lstlisting}
    quex::my_lexer*  qlex = new quex::my_lexer("program.mine", "GREEK-CCITT");
\end{lstlisting}

if one intends to create a purely greek programming language. If you set the command
line option for 'byte per character' {\tt -b wchar\_t}, then {\quex} will use
the number of bytes that corresponds to {\tt wchar\_t} on your platform. For
anything else then ASCII input encodings do not specify a byte per character
less than two.  At the time of this writing {\quex} does not support internal
engines that are not based on Unicode. Clearly, in the above example the turkish
encoding is treated internally as character codes in the range of code points from 
\$370 to \$3FF and \$1F00 to \$1FFF.
