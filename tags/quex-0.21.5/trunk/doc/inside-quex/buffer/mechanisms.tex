The buffer implementation in {\quex} takes advantage of the mechanism of
lexical analysis, and manages to keep the overhead of buffering extremely low.
This works as shown in figure \ref{fig:algorithm}. First of all, some content of the
information stream in stored in a fixed chunk of memory, the buffer. During the
lexical analysis one basically iterates through this chunk. Whenever a new
input character is required to determine a state transition, the pointer to the
current character, short the 'current' pointer, is increased and the input is
assigned the content to what this pointer points. The buffer limits, begin of
file, and end of file is determined through special characters.  Thus, if the
input that is received equals those, then one touched either a buffer limit,
the begin of the file, or the end of the file. The trick here is that the
transition checks of any state will drop if those characters occur. In
any other case the business can continue as usual.

\showpic
{figures/algorzthm.pdf}
{Interaction of buffer and the lexical analyser state machine.}
{fig:algorithm}

The mechanism described above is highly efficient, since there is a {\it zero
overhead} as long as no limit is reached. Only when the buffer limit is reached
some overhead occurs for identifying that the drop out occured either due to
buffer limit or end of file (begin of file, if one iterates backwards during
	pre-conditions).  Already with a buffer size of 64KB, the buffer limit
is only reached every 65536 characters.  With an average of 4 operations per
character the time overhead should not exceed 0.02\% for buffer loading.  The
speed is basically equivalent to iterating directly through system memory.

