Maybe, an even greater contribution than the program itself is that it makes a
strong argument, that even with the rise of PEG parsers lexical analysis and
parsing are two different, separate things. the production of implicit tokens
has nothing to do with grammar and necessitates the abstraction of a lower
level module for lexical analysis. The on- and off-switching of pattern sets
for mini-sub-languages are naturally described by modes of lexical analysis.
Grammars do not support the concept of modes. Finally, a last argument is a
more philosophical one: Interpretation is all about understanding some traces
that are supposed to contain information.  Before one can hope to understand
anything there {\it must be} some {\it signals}, i.e. something in the traces
must have a clear meaning and must be related to some known concept. Otherwise
no content can ever be identified.  In spoken language the words $[$George$]$,
$[$Erwin$]$, $[$beat$]$, and $[$Joe$]$ are signals with a clear and distinct
meaning. Being able to identify them and knowing to what they relate are a
first step to understand a sentence.  Composition, is a powerful tool to leave
the set of signals to be known small while spanning a large space of possible
information. One could, for example, invent a new signal for each message of
the type 'A-guy-named-Joe- beats-a-guy-named-George' and
'A-guy-named-George-beats-a-guy-named-Erwin'.  But, the sheer number of
possible permutations and arrangements will make the set of signals so huge
that it grows beyond the capability of any (carbon- or silicon-based) system
to maintain a codebook of all signals. In contrast, one can define a
composition rule such as 'The first signal is a doer. The second signal is an
action. The third signal is the thing on which the action is applied'. Now, A
sequence of signals $[$Joe$]$ $[$beats$]$ $[$Erwin$]$ or any other combination
can be easily interpreted. Thus, the two step approach of {\it signals which
  are easy to identify} (lexical tokens) and {\it composition} (grammar) is
optimal with respect to the ability to handle complex information. Again, the
linear structure of regular expressions supports the ease of indentification,
where else the recursive nature of grammar supports complexity. Thus, lexical
analysis as the unit for producing signals shall not given the opportunity to
treat recursive patterns. Thus, melting parsing and lexical analysis is
evil\footnote{An exception might be made for tiny quick-and-dirty languages.
  If there is no need for mini-sub languages and implicit tokens, and if the
  language grammar is small enough for not getting confused by lexical
  analysis rules, then there might be an advantage of convinience.}!

%% A final remark on history: {\SaferTeX}, the project in whichs frame {\quex}
%% evolved, is now renamed {\Boox} and tries to implement a next generation
%% typesetting system, i.e. a successor of {\TeX}.

