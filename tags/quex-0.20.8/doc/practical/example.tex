This practical examples guides through the phases of creating a lexical
analyser using {\quex}. Copy pasting this example may be a good starting point
for the work with {\quex}. Figure \ref{fig:process} displays the major actions
to be taken. There are basically two major steps. The first step is a
pre-condition for the second step where pattern-action pairs are defined. This
first step consists of three minor activities:

\showpic
{figures/process}
{Activities for Lexical Analysis Design}
{fig:process}


\begin{enumerate}
\item {\it Definition of Patterns}. Patterns can be defined directly on the
  pattern action pair line, but it is much more intuitive to define patterns
  in terms of regular expressions \cite{} in a separate data file. Those
  patterns act like constants that allow to understand quickly for what a
  pattern/action pair stands. Using a name '{\tt NOT\_NEWLINE}+' is simply
  visually more appealing than writing {\tt $[$\verb|^\|n$]$+}. Those patterns
  can be defined very easily in terms of regular expressions in a dedicated
  pattern file.
  
\item {\it Definition of Modes}. This is best done with a drawing program or
  by hand on paper. One should be clear about how lexer modes are named, how
  they relate to each other and what transitions are possible.
  
\item {\it Definition of Token-IDs}. We finally want to send tokens and those
  tokens must have identifiers. For this reason, the user needs to specify
  names of tokens in a token-id file, so that {\quex} can create ids for them.
  
\end{enumerate}

Once, these simple steps are accomplished the pattern-action pairs can be
coded. The example in this section describes a linear process of these
activities. In practical work, however, this process will most likely be
iterative.


