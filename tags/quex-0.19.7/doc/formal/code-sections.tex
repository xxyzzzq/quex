Previous chapters and sections introduced different code sections where
patterns, token identifier, modes, and engine code could be specified. The
following table summarizes these section in order to serve as an overview. It
associates the code section's keyword with its content and the section and page
reference of its explanation.
\vskip0.5cm
\begin{tabular}{|l|l|l|l|} \hline
{\bf Keyword} & {\bf Task}                                     & {\bf section} & {\bf Page} \\ \hline \hline
{\tt mode}  & Pattern-action pairs, i.e. the 'behavior' of the & \ref{sec:formal/modes} 
                                                               & \pageref{sec:formal/modes}  \\
	    & lexical analyzer.                                & \ref{sec:formal/pattern-action} 
     	                                                       & \pageref{sec:formal/pattern-action} \\ \hline
{\tt define}  & Pattern name defintions, i.e. naming of regular & \ref{sec:practical/patterns}  
                                                                & \pageref{sec:practical/patterns}  \\
	      & expressions to be used in pattern-action pairs. &         &             \\ \hline
{\tt token}   & Definition of token-ids.                       
                                             & \ref{sec:practical/token-ids}  
                                             & \pageref{sec:practical/token-ids}  \\ \hline
{\tt header}  & Code to be pasted above the engine class.      
                                             & \ref{sec:formal/pasting-header}  
				             & \pageref{sec:formal/pasting-header}  \\ \hline
{\tt body}    & Code to be pasted into the engine class definition. 
                                             & \ref{sec:formal/generated-class/class-body-extension}  
                                             & \pageref{sec:formal/generated-class/class-body-extension} \\ \hline
{\tt init}    & Code to be pasted into the engine's constructor.    
                                             & \ref{sec:formal/generated-class/constructor-extension}  
					     & \pageref{sec:formal/generated-class/constructor-extension} \\ \hline
\end{tabular}
\vskip0.5cm
For convinience, even though, it is not a 'region', let it be mentioned here, that he 
assignment 
%%
\begin{lstlisting}
    start = MY_MODE
\end{lstlisting}
%%
defines the starting mode of the lexical analyzer. If no starting mode is explicitly defined
{\quex} assumes the first mode defined in the source code as starting mode and produces a 
warning of the console.
