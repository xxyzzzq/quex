Many programming languages provide an 'include' feature. This means that they
allow the user to get some input from another file and then returning to the
file itself. If in C one specifies 'stdio.h' to be included, then the content
of this file will be treated before one continues parsing the current file.
Such features are popular and provide a lot of convenience to the user. In a
more general sense, what happens is that the input stream changes and needs to
return after the alternative input has finished. This sounds like a case for
the memento pattern \cite{} where the state of the lexical analyzer needs to be
stored and restored---and it is.

The decision of what information about the lexical analyzer is to be stored
during a input channel change, i.e. an 'include' is not left to the user. The
generated analyzer class provides the user with handy functions that do the
whole job. Those are the following:

\begin{lstlisting}
	template <class InputHandle> 
	void   include_stack_push(InputHandle*, const int MODE_ID = -1, 
                              const char* IConvInputCodingName = 0x0);
	template <class InputHandle> 
	void   include_stack_push(InputHandle*, const quex_mode&,
                              const char* IConvInputCodingName = 0x0);

	bool   include_stack_pop();
\end{lstlisting}

The functions can be grouped according to two actions: pushing the current state
of the lexical analyzer on an include stack and popping the last pushed state
of the lexical analyzer from the include stack. The overloaded functions
{\tt include\_stack\_push($\ldots$)} require as a first argument a {\tt FILE*}
or an {\tt istream*} pointer of the stream which is supposed to be the new
input stream. The analyzer then automatically creates a new buffer object and
sets everything up for the input from the new stream. Also, the current
state is stored in a stack. If the program feels that is has finished parsing 
an {\it included file}\footnote{This happens usually on the occasion of 'end of file'.}
then it may want to return to the {\it including file}. The reset of the old 
state is now easily possible using the function {\tt include\_stack\_pop()}. 
It {\it deletes} the buffer of the included file, but it does not delete
the related input stream. If there is something to do with it, it has to be done
{\it before} calling {\tt include\_stack\_pop()}. In particular, it 
is generally a good idea to close the file stream when reading has finished.
