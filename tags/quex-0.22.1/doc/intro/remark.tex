The versioning of {\quex} follows a scheme that is widespread in the
GNU/Linux-Community. The key idea is that version 1.0.0 marks a major step in
the design and development of the software. A {\quex} release with version
greater or equal 1.0 will not be backward compatible (at least to a very high
degree). It says that the groundwork design is frozen, new features can be
added but previous work can be continued with any later version. That,
however, does not mean that versions less than 1.0.0 are not stable or
untested. In fact, at least the author uses {\quex} heavily in another
project. So, the author of this software himself has a vivid interest to
delete any possible bug. 

Also, the second number in the version identifier, i.e. the 'x' in "version
0.x.23" stands for at least one new concept or new design that was
implemented. The last number simply counts releases. From the above
considerations, though, one can conclude that lower last numbers are closer to
design changes and mark 'break-through' versions. 



%%% Local Variables: 
%%% mode: latex
%%% TeX-master: t
%%% End: 
