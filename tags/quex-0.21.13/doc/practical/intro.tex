This practical examples guides through the phases of creating a lexical
analyser using {\quex}. The files of this example can be found in the directory
{\tt DEMO/001/} of the {\quex} distribution. The file {\tt simple.qx} contains
the description of the lexical analyzer, {\tt lexer.cpp} contains a lexical
analyzer application, and a {\tt Makefile} allows the convinient creation of
the whole lexical analyzer application. Copy-pasting this example may be a good
starting point for the work with {\quex}. A good approach is to devide the development
of a lexical analyzer into the two steps of {\it design} and {\it implementation}. 
The first step consists of three minor activities:

%% \showpic
%% {figures/process}
%% {Activities for Lexical Analysis Design}
%% {fig:process}

\begin{enumerate}
\item {\it Definition of Patterns}. Patterns can be defined directly on the
  pattern action pair line, but it is much more intuitive to define patterns
  in terms of regular expressions \cite{} in a separate data section. Those
  patterns act like constants that allow to understand quickly for what a
  pattern/action pair stands. Using a name '{\tt NOT\_NEWLINE}+' is simply
  visually more appealing than writing {\tt $[$\verb|^\|n$]$+}. Those patterns
  can be defined very easily in terms of regular expressions.
  
\item {\it Definition of Modes}. This is best done with a drawing program or
  by hand on paper. One should be clear about how lexer modes are named, how
  they relate to each other and what transitions are possible. This design
  will be the basis for the later coding of modes.
  
\item {\it Definition of Token-IDs}. We finally want to send tokens and those
  tokens must have identifiers. For this reason, the user needs to specify
  names of tokens in a token section, so that {\quex} can create ids for them.
  
\end{enumerate}

Once, these simple steps are accomplished the implementation of modes in terms
of pattern-action pairs can be started.  The example discussed in this section
develops the following files that are required for generating a lexical
analyser:

\begin{description}
  \item[\tt simple.qx] which contains all information {\quex} needs to create
                       a lexical analyzer. 
  \item[\tt lexer.cpp] which uses the created lexical analyser in a
                       mini example application.
\end{description}

To follow the explanations of this chapter the directory {\tt DEMO/001}
is best copied into a user domain for further examination and modifications.

