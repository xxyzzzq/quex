Additionally to the specification of the pattern to be matched 
{\quex} allows to define conditions on the boundary of the pattern.
This happens through pre- and post-conditions. First, the trivial
pre- and post-conditions for begin of line and end of line are discussed.
Then it is shown how to specify whole regular expressions to express
conditions on the surroundings of the pattern to be matched. The traditional
characters to condition begin and end of line are:


\begin{itemize}

\item {\tt \verb|^|R}: an {\tt R}, but only at the beginning of a line. This condition
        holds whenever the scan starts at the beginning of the character stream or
	right after a newline character. This shortcut scans only for a single 
	newline character '\verb|\|' (hexadecimal $0A$) backwards, independent
	on how the particular operating system codes the newline. In this case, 
	there is no harm coming from different conventions of newline.

\item {\tt R\$}: an R, but only at the end of a line or right before the end of the
     file. Note, that the meaning of this shortcut can be adapted according to the 
     target operating system. Some operating systems, such as DOS and Windows, code
     a newline as a sequence '\verb|\|r\verb|\|n' (hexadecimal $0D,\,0A$), i.e.
     as two characters. If you want to use this feature on those systems, you need 
     to specify the {\tt --DOS} option on the command line (or in your makefile).
     Otherwise, \$ will scan only for the newline character $0A$. 

     Note, that for the trivial end-of-line post condition the newline coding
     convention is essential.  If newline is coded as $0D$,\,$0A$ then the
     first $0D$ would discard a pattern that was supposed to be followed by
     $0A$ only.

\end{itemize}

For more sophisticated case 'real' regular expressions can be defined to handle
pre- and post-conditions. Note, that pre- and post-conditions can only appear
at the front and rear of the core pattern. Let {\tt R} be the core regular expression,
{\tt Q} the regular expression of the pre-condition, and {\tt S} the regular expression
for the post-condition.

\begin{itemize}

\item {\tt R/S}: matches an {\tt R}, but only if it is followed by an {\tt S}. If the 
                 pattern matches the input is set to the place where {\tt R}. The part
		 that is matched by {\tt S} is available for the next pattern to be matched.
                 {\tt R} is post-conditioned.
		 Note, that the case where the end of {\tt R} matches the beginning of 
                 {\tt S} cannot be treated by Version 0.9.0\footnote{The reason for this lies in the 
		 nature of state machines. Flex has the exact same problem. To avoid this
		     some type of 'step back from the end of the post-condition' must be
		     implemented.}.
		 
\item {\tt Q/R/}: matches {\tt R} from the current position, but only if it is 
                  preceeded by a {\tt Q}. Practically, this means that {\quex} goes
		  backwards in order to determine if the pre-condition {\tt Q} has matched 
		  and then forward to see if {\tt R} has matched. {\tt R} is pre-conditioned.
		  Note, with pre-conditions there is no trailing context problem as with 
		  post-conditions above.
                  
\item {\tt Q/R/S}: matches {\tt R} from the current position, but only if the preceeding
                   stream matches a {\tt Q} and the following stream matches an {\tt S}.
                   {\tt R} is pre- and post-conditioned.

\end{itemize}
