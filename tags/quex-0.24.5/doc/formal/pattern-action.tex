{\Quex} only allows to specify pattern-action pairs inside modes. As the name
says, a pattern-action pair consists of a pattern that is to be matched and a
code fragment that is executed when the pattern matches. Patterns are best
specified inside the pattern definition sections, so that inside the mode sections
only names of patterns occur. However, one can also use regular expressions
inside the mode definitions (currently no '\kern 1ex' spaces, though). The format for pattern-action pairs is as simple
as can be: 

\begin{enumerate}
\item The {\it pattern} is specified as a regular expression or a pattern name
embraced by curly brackets.
\item Then comes a {\it whitespace} that separates the pattern from the action.
\item Then there must be a {\it curly bracket followed by whitespace} that
  delimits the start of the code fragment of the action. {\Quex} parses then
  until the {\it matching closing curly bracket} that terminates the action.
\end{enumerate}

Examples for pattern-action pairs have been given enough in section
\ref{sec:practical/pattern-action-pairs}, page
\pageref{sec:practical/pattern-action-pairs}. Note, if you want to define a
pattern that contains whitespace, you must define it in the pattern section!
Then use the pattern macro in the pattern action pair.  Otherwise, {\quex}
might interpret the whitespace as 'end-of-pattern'\footnote{This holds at least
  until version 0.10.0.}. There are a few important variables to know for
coding the action for a pattern match:

\begin{enumerate}
\item {\tt self}: Use the reference {\tt self.} + {\tt member\_name} in order to access a member of
  the lexical analyser. This works in any place - also in event handlers.
  Better do not try to use the {\tt this} pointer in order to keep the code
  uniform.  The {\tt self} reference automatically referes to an object of the
  derived class if such exists, and it points to the lexical analyser even if
  the code is pasted into member functions of different objects.
\item {\tt RETURN}: Use the {\tt RETURN} macro to return from a pattern action 
  pair. This ensures a synchronization with the token queue.
\item {\tt Lexeme}: The string that matched the pattern of the current action
  can be accessed through variable {\tt Lexeme}.
\item {\tt LexemeL}: The length of a lexeme  that matched a pattern can be accessed
  through variable {\tt LexemeL}.
\end{enumerate}





