Figure \ref{fig:load-forward} shows the process of loading the buffer with fresh content in forward direction.
Loading data from a device into a buffer is very costy. So, it is a good idea to introduce
some fallback area where the end of the current buffer content is stored at the beginning.
This prevents an immediate load backwards as soon as a character is required from before the
current position. Also, since the current lexeme may be post-treated by some pattern action,
the pointer to the lexeme start also has to lie inside the buffer. The maximum of both values defines the 
fallback border. 
	
\showpic
{figures/load-forward.pdf}
{Loading new contents 'forwards' from the stream into the buffer.}
{fig:load-forward}

The new content from the file is then stored behind this border. If an
end of file occurs, internally an 'end of file pointer' is set to the position in the
buffer where this occurs (most probably before the end of the buffer, but not necessarily).
It is stored in a pointer, because it needs to be moved during backward loading. Running through the
whole buffer searching for EOF would imply a tremendous slow-down.
Also, the end of file character is stored at the position where end of
file occured. This way, the {\tt get\_forward()} function will return
it and the state transition can react on it.

The setting and unsetting of the buffer limit characters and end of
file characters at the borders is handled by the load function. Basically, when a border
of the buffer does not represent a begin or end of file, it is set to the buffer limit code character.
Else, it contains either the begin of file or the end of file character. Note, that the
begin of file character, if it occurs, occurs only at the begin of the buffer. The
end of file character may occur at any position.

\showpic
{figures/load-backward.pdf}
{Loading new contents 'backwards' from the stream into the buffer.}
{fig:load-backward}

Figure \ref{fig:load-backward} shows the process of backward loading. Unlike loading forward not such a large amount
of memory is newly loaded. Note, that backward loading is only necessary if the fallback buffer
was not enough. Thus, it implies some 'abuse of design' that is caught and treated seeminglessly.
However, a basic assumption about lexical analysis is that the main direction is {\tt forward\/}.
Instead of loading a large bunch of memory before the current position, loading backwards only
loads about the first third of the buffer with backwards content from the file. The two thirds
of buffer remain intact. Thus, when lexical analysis finally goes forward again, there is still
much room before a forward loading is necessary.


