Repetition must govern the following scenarios:

\begin{description}
\item[{\tt R$*$}:] repetition of a pattern {\tt R} zero or an arbitrary number of times.
\item[{\tt R$+$}:] repetition of a pattern {\tt R} one or an arbitrary number of times.
\item[{\tt R$\{$x$\}$}:] repetition of a pattern exactly  {\tt x} times.
\item[{\tt R$\{$x,y$\}$}:] repetition of a pattern at least x and at maximum y times.
\item[{\tt R$\{$x,$\}$}:] repetition of a pattern an arbitrary number of times, 
     but at least {\tt x} times.
\item[{\tt R?}:] repetition of a pattern {\tt R} zero or one time.
\end{description}

The first case {\tt R$*$} is the so called 'Kleene-closure' \cite{}. Again, by means
of the $\epsilon$-transition this can be achieved by the following steps:

\begin{enumerate}
\item Create a new initial state and a new terminal state. The new terminal state
      is an acceptance state.

\item Mount the new intial state and the new terminal state via $\epsilon$-transition
      to the old initial state and the old acceptance states. The
      old acceptance states remain acceptance states.
      
\item Connect all acceptance states backwards to the old initial state
      via $\epsilon$ transition. This way the same character sequence can 
      be entered in the state machine again.

\item Connect the new initial state forwards to the new terminal state
      via $\epsilon$-transition.
\end{enumerate}

Figure \ref{fig:pattern-repeat-kleene-closure} shows the example of a
Kleene-Closure for the pattern \pattern{$[$A-Z$]$+":"}, which would match a C-style
label. The other forms of repetitition are closely related to that. The one or
arbitrary repetition \pattern{R$+$} can be achieved by mounting the same state
machine in front of the Kleene closure, ensuring that the pattern has to match
at least once. Similarly minimum repetitions are achieved by mounting the state
machines N times in front of the Kleene-Closure, where N is the minimum number
of times the pattern has to be matched. 

\showpic
{figures/pattern-repeat-kleene-closure.pdf}
{Pattern repetion by Kleene Closure.}
{fig:pattern-repeat-kleene-closure}

Where a maximum number M of repetitions is involved, one cannot rely on the Kleene
Closure anymore. The state machines have to be mounted M-N times in sequence, where
the acceptance states in between remain acceptance states. This way one can be 
sure that any number of matches from N to M corresponds to an acceptance state.

A special case is the case of where zero and a maximum number of repetions are
involved, i.e.  \pattern{R?} and \pattern{R$\{$x,y$\}$} with $x\,=\,0$. 
In this case, simply the initial state is raised to an acceptance state, and then
the usual algorithm for generating a maximum number of repetitions is applied.

