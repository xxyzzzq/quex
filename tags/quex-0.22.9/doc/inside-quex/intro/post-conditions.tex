A post-condition is a condition on the character stream after the 'real'
pattern has matched. When the 'real' pattern notifies a match at position X,
the analyser needs to proceed in the character stream until it
determines whether the subsequent characters match the post-condition.
If the post-condition is met it reports 'acceptance'---however, the
lexical analyser's focus is put back to position X.	
        
\showpic
{figures/post-conditions-patterns.pdf}
{State machine matching a post condition pattern.}
{fig:example-post-condition}

An example of a post-condition is shown in figure
\ref{fig:example-post-condition}.  The rule for this pattern is that when a
number comes in that is followed by a currency, then it has to be reported as
{\it amount}, rather than a number. In this example there are two currencies:
EUR for 'Euro' and USD for 'US-Dollars'. When a number is matched---with or
without a floating point---then we reached the position to which we want to step
back, if it was an {\it amount}. These acceptance states have a dashed inner
circle, to indicate that the input position has to be marked, but we're not
finished yet. Only after one of the currencies \pattern{EUR} or \pattern{USD} appeared,
finally, an acceptance state is reached. But, the following analysis
needs to start at the position, backwards, where the number matched. 

\showpic
{figures/post-conditions-patterns-sequence.pdf}
{Sequence of analysing steps detecting a post-conditioned pattern.}
{fig:example-post-condition-sequence}

Figure \ref{fig:example-post-condition-sequence} displays the sequence of
actions for an incoming character stream.  When the \pattern{12} has matched as
a number the input position '2' is stored but the analysis continues. When
\pattern{USD} has matched it is now determined, that the previous number was
indeed an amount.  A token 'AMOUNT(12)' can be reported. But, but one needs to
go back to the position where the amount ended, i.e. to position '2'. From
there, it detects a currency \pattern{USD} and reports a token
'CURRENCY("USD")'. The same thing happens now with the number \pattern{45.10}
and the currency \pattern{EUR} in the remainder of the stream.

Post-conditions are implemented by setting a 'store-input-position mark' at the
states where the 'real' pattern matches. The end of the post-condition is an
acceptance state. Figure \ref{fig:post-conditions-merged}
depicts the states of the amount/currency pattern webbed into a 
state machine that contains other patterns. The process of combining
and optimizing the state machine needs to make sure, that the information
about saving input positions and acceptance states remains---here indicated
as notes on a flag.

\showpic
{figures/post-conditions-patterns-merged}
{States of a post-conditioned pattern in the combined state machine.}
{fig:post-conditions-merged}
