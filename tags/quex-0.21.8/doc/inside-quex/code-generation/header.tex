Each character that is read from a character stream
triggers a state transition inside the lexical analyser. This
continues until the currently incoming character does not
trigger any further transition. Following cases need to be
distingushed:

\begin{description}
\item[\bf Mismatch] The lexical analyser never reached an acceptance state.
                    No explicitly defined pattern has matched. If a default
		    action is defined than it is executed at this point.

\item[\bf Match] The lexical analyser reached an acceptance state, some time
		 earlier---or is currently in it. The the action of the
		 dominating pattern has to be executed.
\end{description}

If the current state is an acceptance state, then the id of the
winning pattern and the position where it was reached has to be stored. If the
current state represents the beginning of a post-condition, then one needs to
store the current input position. If later on the post-condition finishes,
then one can go back to it.

What pattern actually wins in an acceptance state depends on two 
issues. First, something that can be compiled into the code: the
patterns priviledge. Seconds, something that has to be determined
at run-time: dependence on a pre-condition flag. Note, that information
about a post-conditioned pattern needs not to be stored, since
the path to the acceptance state can only be reached through the
state where the core pattern would have raised 'acceptance'.
The header of the state transition code of a particular acceptance 
state looks like the following: 

\begin{lstlisting}
  QUEX_LABEL__ENTRY_98:
     QUEX_STREAM_TELL(last_acceptance_position); 

     // (*) patterns that start now with the post-condition
     QUEX_STREAM_TELL(last_acceptance_a_input_position);
     QUEX_STREAM_TELL(last_acceptance_b_input_position);
     QUEX_STREAM_TELL(last_acceptance_c_input_position);
     ...

     // (*) patterns that dominate, but under a pre-condition
     if( pre_condition_i_fulfilled_f ) {
	 last_acceptance = i;
     } 
     else if( pre_condition_j_fulfilled_f ) {
	 last_acceptance = k;
     }	 
     else if( pre_condition_k_fulfilled_f ) {
	 last_acceptance = l;
     }	 
     ...
     else { 
         last_acceptance = X;  // the dominating pattern
     } 

\end{lstlisting}

If the state is not an acceptance state, then this header is simply empty.
However, both types of states, acceptance and non-acceptance states share
the following code fragment:

\begin{lstlisting}
     // (*) get a character from the input stream as a trigger
     QUEX_STREAM_GET(input);
     
     // ... it follows: the state transitions ...
\end{lstlisting}

It simply reads the next character from the input stream. It is canned into
a macro, so the user may adapt its underlying procedure to his way of representing
the character stream (e.g. as a buffer, an object of {\tt istream}, etc.).
The code fragment for the state transition has to determine what trigger 
triggered, and therefore what the subsequent state will be.


