When {\quex} creates a lexical analyser it leaves some switches in the code
that can be turned on and off in order to control certain
charactersistics. First of all, {\quex} makes heavy use of {\tt assert}s to
make sure that the code is executed appropriately. These asserts can be turned
off by defining the {\tt NDEBUG} macro.  The following macros are {\quex}
specific and can be specified with the '-D' option of your compiler:

\begin{description}

\item[\tt DEBUG\_QUEX\_MODE\_TRANSITIONS]: If defined, the lexical analyser
engine displays any mode transition on the standard error output. 

\item[\tt DEBUG\_QUEX\_PATTERN\_MATCHES]: If defined every pattern match is
  going to be displayed on the standard error output. Note, that if the
  lexical analyser is created with option '{\tt --debug no}' (see section
  \ref{sec:formal/command-line-options}), then the source code does not
  contain the code fragments that activate such a printout.

  
\item[\tt DEBUG\_QUEX\_TOKEN\_SEND]: If defined, then every token being sent
  using the {\tt send} function group is displayed on standard error output.
  
\item[\tt QUEX\_FOREIGN\_TOKEN\_ID\_DEFINITION]: If defined the definition of
  token-ids, i.e. their numeric values is taken from an external file.  This
  file needs to be specified using the command line option {\tt
    --foreign-token-id-file}. Foreign token-id definitions make sense, for
  example, if the parser generator insists on defining token-ids.  If this
  macro is not defined the given external file is ignored and
  {\quex}-generated numerical values are used.

        
\item[\tt QUEX\_NO\_SUPPORT\_FOR\_COLUMN\_NUMBER\_COUNTING]: 
  
\item[\tt QUEX\_NO\_SUPPORT\_FOR\_LINE\_NUMBER\_COUNTING]: By defining these
  macros the user can turn off the column number and line number counting. The
  analyser may run a little faster then. Note, that in this case the member
  functions for column number and line number access are no longer available.
        
\end{description}
