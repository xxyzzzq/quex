The lexical analyser is traditionally implemented using an approach of a
finite automaton \cite{} based on regular expressions \cite{}, thus not
allowing {\it recursion of sub-patterns} inside patterns.  Such an automaton
simply sends particular token, when a particular pattern matches. Grammars, in
contrast, do allow recursion. So, the ability of recursion was considered as a
major difference between lexical analysis and parsing. However, 
there are approaches to express lexical patterns with grammar \cite{ANTLR}.
%% recently the re-discovery of {\it parsing expression grammars}
%% \cite{Birn-guy:1970} \cite{Ford:04} made it possible to overcome this
%% restriction.
%% Together with their ability to describe grammars more effectively than
%% traditional {\it context free grammars}\cite{} and {\it context sensitive
%%  grammars}\cite{}.
In general, grammars are a stronger tool to describe patterns than regular
expressions, because they do allow recursive sub-patterns.  The author of this
text, though, says: "This is an evil development!". First of all, recursion is
a means to model sub-blocks of information or to present parenting
relationships. This is in total contrast to the idea an atomic information
chunk as basis for a grammar, since atoms should be inseparable and should
not be some higher level construct. But, there are more very strong arguments
for not melting the lexical analyser with the parser.

Oringinaly, the {\quex} system was developped inside a interpreter called
{\SaferTeX}\cite{TUG-2004}. The goal was to produce a beautiful programming
language for typesetting. The beauty of its source code appearance was to be
achieved through redundancy reduction, i.e. the programmer only should provide
information that is absolutely necessary, thus avoiding visual noise. In
{\LaTeX}, for example, an item list is specified as follows:

\begin{lstlisting}
\begin{itemize}
\item This is the first item ...
\item This is the second item ...
...
\end{itemize}
\end{lstlisting}

An sophisticated lexical analyser, though, should be able to recognize that
when an {\tt \verb|\|item} token arrives there must be an item list that
starts, if it is not already started. It should notice, that a token {\tt
  \verb|\|begin\{itemize\}} has to be implicitly be sent. Also, indentation
based language, such as Python, require the production of {\it implicit
  tokens} for block closure when lower indentations occur. Since implicit
tokens build an important pillar for redundancy reduction, the lexical
analyser system of {\SaferTeX} was therefore designed for the support of {\it
  implicit tokens}. 

Another issue are {\it modes of lexical analysis}. For example, while
analysing a format string, it does not make sense to scan for mathematical
patterns such as '+', '-', 'sin', and 'cos'. When analysing a format string
one expects patterns of the type '\%s', '\%02x', '\%7.1f' or so. Different
subjects such as string formatting, mathematical algorithms, or compile time
computation are best described by a dedicated mini-sub-language. Such
dedicated languages shall allow the presentation of information with minimum
usage of language constructions. Each a mini-sub-language may have its own
bank of patterns to be recognized. On the other hand, some patterns such as
'end-of-file', 'comment-start', and 'comment-end' shall exist in all
mini-sub-language and produce exactly the same tokens. Lexical analyser modes
that can have inheritance relationships are a powerful means to mode this.

In grammars, sub-languages are tamed by 'bracket' tokens and {\it not by
  parser modes}. If it is desired to have a mini-graph-plotting language, for
example inside a main language, one would expect a grammer like this:

\begin{lstlisting}
plot_program = PLOT_BEGIN plot_statement_list PLOT_END

plot_statement_list = plot_statement
                    = plot_statement_list  plot_statement

plot_statement  = 'plot' plot_range file_name 

plot_range = '[' number ':' number ']' 
\end{lstlisting}

Lexical mode transitions and mode inheritance relations have become key
concepts for the lexical analyser of the {\SaferTeX} system. Since these
features seemed to be unique to the author, an effort was done to extract the
lexical analyser from {\SaferTeX} into an independent program {\quex}, and
make it available to a broader public. Together, with a PEG parser\cite{}, it
is now possible to parse even language such as Python, which cannot not be
parsed without implicit tokens. {\quex}, therefore, {\it extends} the space of
parsable languages.

