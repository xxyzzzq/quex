In order to implement the logical 'or' operation in regular expressions it must
be possible to mount in parallel.  Therefore, all acceptance states of the
original state machines to be mounted in parallel need to remain acceptance
states. All initial states need to remain initial states. However, there can be
only one initial state. At this point the $\epsilon$-transition comes handy
again. The following steps describe the process of mounting state machines in
parallel:

\begin{enumerate}
\item Clone both state machines (see \ref{sec:thomson-construction/serial}).
\item Mount an epsilon transition from all acceptance states of all 
      state machines
      to a new terminal state T. Let T be an acceptance state.
\item Disable original acceptance of all acceptance states in all state machines.
\item Create a new initial state I and mount an $\epsilon$-transition to 
      all initial states of all state machines.
\end{enumerate}

Figure \ref{fig:mount-parallel} shows the result of combining three state
machines in parallel. By means of this procedure there is only one initial
state and one terminal acceptance state for the state machine. Now, regular
expression such as \pattern{"hello"$|$"bonjour"} can be implemented as two parallel
state machines for \pattern{"hello"} and \pattern{"bonjour"}.

\showpic
{figures/pattern-parallelize.pdf}
{Mounting two state machines in parallel.}
{fig:mount-parallel}

