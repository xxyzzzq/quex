With a similar ease as patterns, token-ids can be specified. A {\tt token}-section  containing
token-ids only has to contain a newline separated list of token names.
{\Quex} will create a set of constants with unique numerical values. Figure
\ref{fig:token-ids-code} shows the definition of token-ids for the example.

\begin{figure}
\begin{lstlisting}
token {
    BRACKET_O
    BRACKET_C
    CURLY_BRACKET_O
    CURLY_BRACKET_C
    OP_ASSIGNMENT
    IF
    STRUCT
    SEMICOLON
    IDENTIFIER
    NUMBER
    STRING
}
\end{lstlisting}
\label{fig:token-ids-code}
\caption{Definition of token-ids in a {\tt token}-section.}
\end{figure}

Note, that {\quex} pastes a prefix in front of the numerical constants for the tokens.
A token-id of {\tt STRUCT} will be called {\tt TOK\_STRUCT}, if {\tt TOK} is
specified as token-prefix. This is discussed briefly in the section
\ref{sec:formal/command-line-options}. It is also important to remind that
the token-ids {\it are} relevant to the namespace. They name constant variables
inside the namespace {\tt quex}. It is not advisable to set the token-id prefix
to an empty string and name a token-id '{\tt i}' or '{\tt x}' because those 
names are likely to interfer with names of other variables.

