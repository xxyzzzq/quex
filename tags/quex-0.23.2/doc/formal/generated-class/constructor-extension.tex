The last section discussed how additional contents can be pasted into the
defintion of the class' body that represents the lexical analyser. Naturally,
those new contents need to be initialized. This happens inside the constructor
of the generated lexical analyser. {\Quex} allows to specify additional
content to be executed inside the constructor using the {\tt init} keyword
outside any mode definition. The following code fragment shows a setup as
it might occur in a lexical analyser that supports indentation based 
statement blocks:

\begin{lstlisting}
body {
    std::vector<int>  indentation_stack;
    bool              allow_opening_indentation_f;
}

init {
    // first indentation at column = 0
    self.indentation_stack.push_back(0);
    // default: do not allow to open a sub-block.
    // only function definitions, if statements, and for loops
    // shoul allow to open a new indentation block in the next line.
    self.allow_opening_indentation_f = false;
}
\end{lstlisting}

The {\tt body} fragment adds an indentation stack as a member to keep track of
line indentations. A second member variable tells if a higher indentation is
supposed to appear that opens a statement block. These two variables need to
be initialized. Therefore, the {\tt init} fragment contains an initial block
starting at column '0'. This indentation block cannot be closed (except by
'end-of-file') because no column can have a negative indentation. Also, by
default open sub-blocks is disallowed, so the correspondent member variable is
set to {\tt false}. This code fragment will then appear in any constructor
of the generated lexical analyser class.


