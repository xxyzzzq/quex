 'make' program
basically works on three concepts: targets, dependencies, and build
procedures. For example the makefile code:
\begin{lstlisting}

target.o: dependency_0.cpp dependency_1.h

        g++ -c -I./ dependency_0.cpp -o target.o

\end{lstlisting}
says that the target, i.e. the object file '{\tt target.o}', depends on the
must recent versions of '{\tt dependency_0.cpp}' and '{\tt my_header.h}'.  If
any of those two changes the target has to be build again. The build procedure
is then given in the following lines, here the compilation of the source file
'{\tt dependency_0.cpp}. In the Makefile of the example lexical analyser, the
following three main dependencies exist:
