The purpose of this chapter is to show a tiny application that does use
lexical analysis in order to implement a very tiny scripting language. This
scripting language allows to set variables, print their content and compute
the sinus and cosinus of numbers. There are mainly four sections. First,
patterns are defined. Second, token names are specified. Third, it is
determined what patterns produce what tokens. Forth, a little programs needs
to be discussed that uses the produced lexical analyser and implements the
minimalist scripting language. All source code related to this example can be
found in the directory \$(QUEX\_PATH){\tt /TEST/000/}.  This directory is best
copied to a user directory for this introduction.

So, first, let us define a pattern file. In the quick-start example most
patterns are defined directly in the mode definition. However, pattern
definitions inside modes cannot contain whitespace, so we need to 
define a pattern macro for that. Let us store 

\begin{lstlisting}
  P_WHITESPACE  [ \t\n]
\end{lstlisting}

in a file {\tt PATTERN.lex}.


\begin{lstlisting}
  mode ONE_AND_ONLY { 

    <<EOF>> => quex::token::ID_TERMINATION;

    "set"       => TKN_TIMEOUT;
    "print"     => TKN_ADDRESS;
    "sin"       => TKN_SEND;
    "cos"       => TKN_SEND;
    ";"         => TKN_SEMICOLON;
    [0-9]+      => TKN_NUMBER(atoi(yytext));
    [_a-zA-Z]+  => TKN_IDENTIFIER(yytext);
    {P_WHITESPACE} { }
  }
\end{lstlisting}


\begin{lstlisting}
TIMEOUT;
ADDRESS;
SEND;
EXPECT;
SEMICOLON;
NUMBER;
IDENTIFIER;
\end{lstlisting}