Any compiler or lexical analyzer imposes rules on the text that is has to
treat.  Most of thoses texts are written by humans and humans make sometimes
errors and disrespect rules.  A gentle compiler tells its user about his
errors and tells it also about the place where the error occured. Here line
numbering becomes crucial. Additionally columns can be counted. This provides
even more detailed information. 


There is general default algorithm to count lines and columns which is always applied 
in case that there is no better alternative. {\Quex} analyzes the patterns 
and adapts the way of counting lines and
columns according to special characteristics\footnote{Actually, even the indentation count algorithm is adapted to
    profit from knowledge about the patterns internal structure.}. If the
    pattern contains for example a fixed number of newlines, than only a fixed
    number is added and no newlines are counted at runtime. The mechanisms for line and column counting are optimized
    for the most 'reasonable' pattern characteristics. There are strange
    cases\footnote{Example: A pattern that contains a newline which is followed
        by a fixed number of characters. The determination of this in the
            context of post-conditions is complicated. On the other hand, such
            patterns are considered strange and seldom, so the expected gain
            with an optimized algorithm was considered neglectable by the
            author.}, though, for which a slightly better counting mechanism might be
            found. For reasonable applications as known from popular programming languages
            the mechanisms of counting should be optimal.

Note, that line and column counting can be turned off individually by pre-processor
switches. By these switches the generated analyzer is 'naked' of any counting mechanism
and it is likely that it runs a little faster. For serious applications, though, at least line number
counting should be in place for error reporting.
