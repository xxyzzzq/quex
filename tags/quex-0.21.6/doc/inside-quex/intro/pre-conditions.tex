A Pre-Condition is a condition on the character stream before the pattern is
actually being parsed. Starting from the current position X, the analyser needs
to go backwards to check whether the pre-condition is
met.  After determining whether it is met or not, the 'real' pattern is
analysed starting from position X.
	  
Intuitively, it does not make sense to try to match against a pattern if it is
known from the beginning that the pre-condition is not met. For this reason, it is tempting to
consider a single analyzer function for each case of 'pre-condition met' and
'pre-condition not met'.  With this approach, though, the total number of
analysers to deal with $N$ pre-conditions is $2^N$ since any combination of
'pre-condition met' and 'pre-condition not met' has to be considered. Very few
pre-conditions would then blow the number of analysers beyond what is
practical. {\Quex} could not be a general solution for pre-conditions.

Rather than creating an analyser for each pre-condition constallation, one
single analyser containing all pre-conditioned and non-pre-conditioned patterns
is created. Pre-conditions are simply implemented by a conditional acceptance
at the end of the pattern, i.e.

\begin{lstlisting}
    // acceptance state ...
    if( pre_condition_X_fulfilled_f ) last_acceptance = X;
    ...
\end{lstlisting}

When the analyser reaches the acceptance state of a pre-conditioned pattern it
notifies a match only under the condition that the pre-condition flag was
raised before. From one point of view, this is inefficient since in some cases the analyser tries 
to match against a pattern, even though the pre-condition disallows an acceptance. 
On the other hand, the states of the pre-conditioned patterns
are webbed altogether into one single state machine and the overhead for doing
so is minimal. 

\TODO{It would be an improvement to allow the user control when a 
    pre-conditioned pattern constallation is webbed into the combined
	state machine and when not.}

