Section \ref{sec:practical/patterns} already discussed the format of the
pattern file. In this section, it is described how to specify patterns by the
use of regular expressions. The first releases of {\quex} relied solely on a
core that was produced by flex. From version 0.10.0 on {\quex} provides its own
core engine---leaving the flex engine as an option. For the sake of established
habits the descriptions of regular expressions are kept mostly conform with the
world of lex/flex.  However, {\quex} provides features that flex does not. If
it is intended to use flex as a core generating engine on the long run, then please refer to
the flex manual \cite{}, section 'Patterns'. This section discusses pure
{\quex} syntax. The explanation is divided into the consideration of
context-free expressions and context-dependent expressions.


     
