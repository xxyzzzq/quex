In {\Quex} modes can be changed by two basic mechanisms. First, modes can
simple be {\it changed}. This makes sense, if the new mode knows exactly to what
modes it has to change and this does not depend on the previous mode. Another
mechanism is that of {\it pushing and popping}. This makes sense, if the new
mode shall return to the previous mode. It is therefore similar to a
'call-subroutine' in procedural programming languages. 

Figure \ref{fig:mode-transitions} shows a typical mode transition. Mode X has some patterns
and one of them matched. Inside the pattern match action a transition to mode
Y is requested. This triggers the event handler {\tt on\_exit()} of mode X
before any transition is made. Here the user can send tokens, change some
internal variables or write some debug output. Then 
the event handler {\tt on\_entry()} of entered mode Y is called. Again the user
can provide some actions related to the entrance of that mode.

\showpic
{figures/mode-transition.png}
{Event handling during lexical mode transitions.}
{fig:mode-transitions}

The ability to define event handlers for entering and exiting modes not only
supports implicit token production, but also supports transparency. It can now
exactly be traced how modes are transited. In order to {\it restrict} the
transition from and to lexical modes, {\quex} provides transition control
mechanisms. In the header of a definition of a mode X it can be specified from
what modes the mode X can be entered and to what modes the mode X can exit.
If a mode transition happens that is not conform to this restrictions a
run-time error is triggered. This helps to prevent the lexical analyser to
slide from one mode to another due to some thoughtlessly designed patterns.

Figure \ref{fig:mode-transition-control} shows an example, where mode
transition control comes very handy. Here, there is a lexical mode for parsing
the content of a {\sf FUNCTION}, a mode for {\sf COMMENT} and one for {\sf
  DOCU}-mentation parsing, a mode for parsing a character {\sf STRING} and one
for {\sf MATH}. The design here says, that the comment mode can only be
activated from the {\sf FUNCTION} mode and can only leave to {\sf FUNCTION}
mode. Similarly for the {\sf STRING} and the {\sf MATH} mode. {\Quex} provides
the feature, that for {\sf COMMENT} mode, for example, one can specify that it
can only be entered from {\sf FUNCTION} mode, thus preventing the {\sf DOCU},
{\sf STRING} and the {\sf MATH} mode from ever transiting into {\sf COMMENT}
mode. Also, it can be specified that the {\sf COMMENT} mode can only exit to
{\sf FUNCTION} or {\sf DOCU} mode, thus preventing it from dropping into {\sf
  STRING} or {\sf MATH} mode.

\showpic
{figures/mode-transition-control.png}
{Restricted mode transitions.}
{fig:mode-transition-control}


Extensive mode transitions without a control mechanism as the one implemented
is {\quex} is highly error prone. The advantage here is that the restrictions
on the mode transitions are {\it explicitly\/} specified in the description of
the lexical analyser. The alternative would be to create an external document
and review the behavior of the lexical analyser with respect to mode
transitions whenever an error is assumed from being caused by straying mode
transitions. Practically, solid handling of extensive mode transitions in
lexical analysis are not possible except with a control mechanism as provided
by {\quex}.

