With the files from directory \$(QUEX\_PATH){\tt demo/001/} the creation of the
example lexical analyser application is as simple as can be. Type {\tt make}
in the command line and the application is built. The Makefile that comes with
the sample application is written in a way that makes it very easy to adapt
and extend its contents for other user's needs. The core of this make file 
explains how to call {\quex}:

\begin{samepage}
\begin{lstlisting}
# (*) definitions
MODE_FILES  = simple.qx
#
ENGINE_NAME = tiny_lexer
#
ENGINE_SOURCES = $(ENGINE_NAME)             \
                 $(ENGINE_NAME).cpp         \
                 $(ENGINE_NAME)-internal.h  \
                 $(ENGINE_NAME)-token_ids   \
                 $(ENGINE_NAME)-core-engine.cpp
...

# (*) Quex builds source code for lexical analyser
$(ENGINE_SOURCES): $(PATTERN_FILE) $(MODE_FILES) $(TOKEN_DB)
        quex --mode-files $(MODE_FILES)      \
             --engine     $(ENGINE_NAME)

my_lexer: 
\end{lstlisting}
\end{samepage}

The build rule says that whenever one of the files containing mode descriptions
changes the source code for the lexical analyser has to be rebuild. In our
example the only mode file is {\tt simple.qx}. {\Quex} receives the names of the 
input file names as 'no-minus' followers of the command line option 
'{\tt --mode-files}'.

The output of {\quex} is determined by the option {\tt --engine}. The name
that follows it specifies the name of the class that implements the engine.
But, it also acts as filestem for all output files. In our example, the
engine's name is {\tt tiny\_lexer}. Therefore, there will be four files
created:

\begin{description}
\item[File: \tt tiny\_lexer] containing the header file that contains the class
  definition of class {\tt quex::tiny\_lexer}. This is {\it the only output file
  that is of direct interest to the user}. It has to be included in each file
  that interacts with the produced lexical analyser.

\item[File: \tt tiny\_lexer.cpp] containing the created lexical analyser
  engine. The user does not need to touch it, but it has to be compiled
  sometime and linked to the application. The example Makefile does this
  already. {\it The user is not directly concerned with this file}.
  
\item[File: \tt tiny\_lexer-token-ids] containing definitions of token-ids.
  That means it provides variables with the names of token-ids that carry
  unique numerical values. It also defines the member function
  
  \begin{lstlisting}
        string token::map_id_to_name(const id_type TokenID)
  \end{lstlisting}
  which maps a token-id to a token-id name. This comes practical in many
  stages of the development. However, this file is included inside the header
  file for the lexer class definition and {\it the user does not worry
    about this file at all}.
  
\item[File (only with flex): \tt tiny\_lexer-internal] is a C-style header file containing
  extracted information from a file that was initially created by flex. The
  user is not concerned of this file.

\item[File (only with {\quex}-core engine): \tt tiny-lexer-core-engine.cpp] is
  a generated C++ code file containing the generated lexical analyzer engines
  for all modes.  
\end{description}

Once the engine sources are built by {\quex} (and possibly by the help of flex), an
example application can be built that uses the lexical analyser.  The source
code for the example is specified in file '{\tt lexer.cpp}'. The content
of this file is shown in figure \ref{ref:sample-application}.

---------------------------------------------------------------------------------------
#include<fstream>    
#include<iostream> 

// (*) include lexical analyser header
#include <./tiny_lexer>

using namespace std;

int 
main(int argc, char** argv) 
{        
    // (*) create token
    quex::token        Token;
    // (*) create the lexical analyser
    //     if no command line argument is specified user file 'example.txt'
    quex::tiny_lexer*  qlex = new quex::tiny_lexer("example.txt");

    // (*) set the initial mode of the lexical analyser
    qlex->set_mode_brutally(qlex->PROGRAM);

    // (*) print the version 
    cout << qlex->version() << endl << endl;

    cout << ",-----------------------------------------------------------------\n";
    cout << "| [START]\n";

    int number_of_tokens = 0;
    // (*) loop until the 'termination' token arrives
    do {
        // (*) get next token from the token stream
        qlex->get_token(&Token);

        // (*) print out token information
        //     -- line number and column number
        cout << "(" << qlex->line_number() << ", ";
        cout <<        qlex->column_number() << ")  \t";
        //     -- name of the token
        cout << Token.type_id_name() << endl;

        ++number_of_tokens;

        // (*) check against 'termination'
    } while( Token.type_id() != quex::token::ID_TERMINATION );

    cout << "| [END] number of token = " << number_of_tokens << "\n";
    cout << "`-----------------------------------------------------------------\n";

    return 0;
}
\end{lstlisting}
\caption{A sample application using the generated lexical analyser {\tt
    tiny\_lexer}.}
\label{ref:sample-application}.
---------------------------------------------------------------------------------------

Initially, the lexical analyser needs to be created as an object of type {\tt
  quex::tiny\_lexer}. The filename of the file to be analysed is directly
passed to the constructor of the class. Then, one needs to set the initial
lexical analysis mode. In our example, {\tt PROGRAM} shall be the initial
mode. Since it shall not trigger any mode transitions or protection
algorithms, it is set using the function {\tt set\_mode\_brutally()}. 

After printing out the version information, the loop over the token stream
starts. In the sample application it does not more than getting the a token
using {\tt get\_token()} and printing it out. In a more realistic application
this token is to be passed to a parser that does its syntactic analysis on it.
When a token arrives with an identifier equal to {\tt
  quex::token::ID\_TERMINATION}, then the loop exits and the program
terminates.

At this point in time everything is ready for compilation. Using the Makefile
coming with the example, typing {\tt make} shall produce a ready-to-run
application called {\tt lexer}  in the current directory. If one types now

\begin{verbatim}
     > ./lexer example.txt
\end{verbatim}

on the command line the following output shall be produced:

-----------------------------------------------------------------------
\begin{samepage}
\begin{verbatim}
tiny_lexer: Version 0.0.0-pre-release. Date Sat Jul  1 12:00:20 2006
Generated by Quex 0.0.1

,-----------------------------------------------------------------
| [START]
(0, 0)          STRUCT
(0, 7)          IDENTIFIER
(0, 15)         CURLY_BRACKET_O
(1, 2)          IDENTIFIER
(1, 9)          IDENTIFIER
(1, 10)         SEMICOLON
(2, 2)          IDENTIFIER
(2, 9)          IDENTIFIER
(2, 10)         SEMICOLON
(3, 0)          CURLY_BRACKET_C
(3, 1)          SEMICOLON
(5, 0)          IF
(5, 3)          IDENTIFIER
(5, 12)         CURLY_BRACKET_O
(6, 2)          IDENTIFIER
(6, 7)          IDENTIFIER
(6, 14)         NUMBER
(6, 34)         STRING
(6, 35)         SEMICOLON
(7, 0)          CURLY_BRACKET_C
(7, 1)          <TERMINATION>
| [END] number of token = 21
`-----------------------------------------------------------------
\end{verbatim}
-----------------------------------------------------------------------

%%% Local Variables: 
%%% mode: latex
%%% TeX-master: t
%%% End: 
