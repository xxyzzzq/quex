The name {\quex} has obviously a lexo-graphical relation to lex and flex---the most
known tools for generating lexical analysers at the time of this writing.  The
'Qu' at the beginning of the word stands for 'quick', so hopefully the new
features of {\quex} and the ability to created directly coded engines will
generate much faster lexical analysers. Also, the process of programming shall
be much quicker by means of handy shortcuts that quex provides and the elegance that
problems can be solved. 

The last letter in the name '$\chi$' is a the greek lowercase letter 'chi'.  It
is intended to remind the author of this text that he actually wanted to create
a better \TeX\footnote{As Donald Knuth explains it\cite{}, the last letter in
\TeX is an uppercase 'chi' so the name is to be pronounced like the German
word 'ach' (it's actually more a sound Germans utter, rather than a word with 
a dedicated meaning). Analogously, the $\chi$ at the end of {\quex} should not 
be pronounced like the 'x' of the German word 'nix'.} system with optimized source code appearance. When he
realized that traditional lexical analysers did not provide the
functionality to express solutions easily he started working on
{\quex}---and dived into the subject much deeper than he ever thought he would. 



