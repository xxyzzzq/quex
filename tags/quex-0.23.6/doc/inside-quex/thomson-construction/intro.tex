The lexical analyser that {\quex} creates is based on regular expressions.  In
order to create state machines that represents regular expressions, it is
necessary to provide operations that combine elementary regular expressions
through {\it concatenation}, {\it parallelization}, {\it optionality} and {\it
    repetition}. This section discusses how this is accomplished by means
    of the so called {\it Thomson Construction} \cite{}.



An essential concept is the so called $\epsilon$-transition. By means of this
special transitions state machines can be glued in sequence and in parallel. It
builds a very basic concept in the Thomson construction for developing a state
machine that represents a regular expression.

\definition{{\bf An $\epsilon$-transition} is a transition that does not
    require any input. If a state A is connect with a state B via an
	$\epsilon$-transition, then entering state A implies that the state
	machine enters state B.}

In a sense, an $\epsilon$-transition behaves like a 'free-ride' where no input character
is necessary to trigger a subsequent state. However, those transitions imply that the 
state machine become non-deterministic, since it can now be in multiple states at the same time.

A very important principle which facilitated writing algorithms on state machines
was that every state-id in every state machine is {\it  unique\/}. That is, 
a state machine A cannot have a state-id that also appears in B. Further,
state machines are mere {\it collections} of states. States are mere {\it mappings}
that describe what subsequent state is reached by what trigger. Operations
on state machines can then be applied by setting transitions, i.e. changing
the mappings of states, and finally collecting all mappings (i.e. states) in
a single state machine. The uniqueness of state-ids ensures consistency.

The following sections discuss the elementary operations on state machines
that are necessary to describe regular expressions. 
