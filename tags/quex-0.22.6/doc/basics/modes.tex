In many cases the behavior of the lexical analysers can be efficiently
described by means of modes for lexical analysis. A standard example are
format strings. Inside format strings, there is no need for the lexical
patterns that are necessary when parsing an algorithm. They are actually
better turned off. Thus, a mechanism is needed that is able to control the
activation and deactivation of groups of patterns to be matched.

A lexical analyser mode is a state of the lexical analyser where only a
particular set of patterns is actively waiting to be matched against.  A
lexical analyser needs to be in exactly one mode at a particular point in
time. Thus, a lexical analyser mode corresponds to the way that the
input stream is currently transformed into signals, i.e. tokens. To handle
the detection and influencing of mode transitions, {\Quex} provides
event handling mechanisms as described in the following sections.

A potential disadvantages of modes is confusion. With traditional lexical
analyser generators, such as flex, the end-user does not see more than a token
stream. He has no inside on the current lexical analyser modes. He cannot
sense or control the mode transitions which are currently made. The mode
transitions are somwhere hidden in the pattern match actions.  GNU Flex's {\it
start conditions} are similar to modes. But, the only way two modes A and B
modes can be related in flex is by letting a pattern be active in in A and
B. There is no convienient mechanism to say: 'let B overtake all patterns of A'.
This is where mode inheritance relationships of {\Quex} provide clear 
convenience advantages as described in the section after the following.


