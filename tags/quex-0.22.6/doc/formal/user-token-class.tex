{\Quex} provides a token-class that allows the storage of a {\tt string}
object, an {\tt int} value. Note, that any complex structure might be
transformed later on by the parser based on the given string. In some cases,
though, it may be promising to implement a dedicated token class that is
optimal in memory and speed with respect to a specific problem. For these
cases, {\quex} allows the user to specify his own token class. The token queue 
does not require any adaption, since it is implemented as a template. The 
context of usages of tokens, however, imposes that it complies to some policies:


\begin{enumerate}
  \item The token class must be specified in namespace {\tt quex}.
    
  \item A {\tt typedef} inside the class must define a type for {\tt
      id\_type}.

    \item The following token-ids need to be defined inside the class:
      \begin{description}
        \item[\tt ID\_UNINITIALIZED] which is preferably set to '-1' so that 
          no interference can occur with positive valued token-ids.
        \item[\tt ID\_TERMINATION] which is preferably set to '0' because many parser
          generators require a token-id of zero to terminate.
      \end{description}

    \item A member function that maps token-ids to token-names
      \begin{verbatim}
        static const std::string&  map_id_to_name(token::id_type);
      \end{verbatim}
      that maps any token-id to a human readable string. Note, that {\quex}
      does generate this function automatically, as long as it is not told not
      to do so by specifying command line option
        \begin{verbatim}
         --user-token-id-file 'filename'
        \end{verbatim}
      
     \item Member functions that set token content, e.g.
      \begin{verbatim}
         void  set(token::id_type TokenID, const char*);
         void  set(token::id_type TokenID, int, int);
         void  set(token::id_type TokenID, double);
         void  set(token::id_type TokenID, double, my_type&);
      \end{verbatim}
       As soon as the user defines those functions, the interface
       for sending those tokens  
       from the lexer is also in place. The magic of templates
       lets the generated lexer class provide an interface
       for sending of tokens that is equivalent to the 
       following function definitions:
      \begin{verbatim}
         void  send(token::id_type TokenID, const char*);
         void  send(token::id_type TokenID, int, int);
         void  send(token::id_type TokenID, double);
         void  send(token::id_type TokenID, int, my_type&);
      \end{verbatim}
       Thus, inside the pattern action pairs one can send tokens,
       for example using the {\tt self} reference the following way:
       \begin{lstlisting}
  {MY_PATTERN} {
        // map lexeme to my_type-object
        my_type tmp(split(Lexeme, ":"), LexemeL); 
        self.send(LexemeL, tmp);
        RETURN;
  }
       \end{lstlisting}

\end{enumerate}

As long as these conventions are respected the user created token class will
interoperate with the framework smoothly. The inner structure of the token
class can be freely implemented according to the programmer's optimization
concepts. Note, that the name of the token class is also of free choice.  When
invoquing {\quex}, the command line option {\tt --token-class} needs to be
followed by the user defined token class name. The command line option {\tt
  --token-class-file} tell {\quex} the name of the while where this class is
defined. As long as this option are not defined {\quex} will not consider
user defined token classes and provide the standard token class.


%%% Local Variables: 
%%% mode: latex
%%% TeX-master: t
%%% End: 
