{\Quex}s buffer is implemented as a template. By this means a maximum of
flexibility is achieved. The following parameters can be passed to the 
template:

\begin{enumerate}
\item {\tt int BOFC}: The character code to designate 'begin of file'.
\item {\tt int EOFC}: The character code to designate 'end of file'.
\item {\tt int BLC}: The character code to designate 'buffer limit'.
\item {\tt class OverflowPolicy}: A class implementing two functions: '{\tt
    forward}' and '{\tt backward}' to deal with the unlike event of buffer
    overflow (see section \ref{sec:buffer/overflow}).
\end{enumerate}

Those are the parameters to the class {\tt basic\_buffer}. The header, though,
defines a default setting for the class {\tt buffer}. If these default do
not interfere with some abnormal goals (such as using EOF inside, not at the border, of a regular
	expression), then these values can be adapted. The easiest way to to so
is to adapt the '{\tt typedef}' statement in the header file:

\begin{lstlisting}
    typedef basic_buffer<0x1, 0x2, 0x0, MyOverflowPolicy> buffer;
\end{lstlisting}

Then, again {\tt quex::buffer} can be used to name the buffer class at any given
place. Constructing a buffer is possible with two basic approaches. The first approach
is suitable for systems where dynamic memory allocation is not an issue. It leaves
the buffer allocation to the constructor of {\tt buffer}. The second approach 
can be used for systems where dynamic memory allocation is an issue, such as
for embedded systems. In these systems memory is a costy resource and there is
no room for slow memory management routines. In this case a pointer to a location
in memory can be passed together with the number of bytes that can be taken
from it.
