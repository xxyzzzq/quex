The previous sections discused the construction of an 'annotated'
state machine representing the lexical analyser. The final step
of code generation remains. Code generation means to write a program
that behaves like the state machine. This means, that if a character $\alpha$
appears at the input one has to transit to the same state as the state machine
would. Further, if the state machine fails, the program needs to fail. And,
    if it succeeds, i.e. a pattern matches, then the program need to notify 
    a match. Additionally, some framework needs to be created for the 
    lexical analyser to iterate over an incoming character stream. 
    
    
    One particular problem has not been discussed before. 'Real' lexical
    analyser search for the longest match, i.e. they try to eat as many
    characters as possible to achieve a match. It is now conceivable that a
    pattern A has matched but pattern B is still in the race and has still hope
    to be matched -- it might only need some more characters. Thus the state
    machine leaves the acceptance state, but {\it it needs to store
	information} about the match of A, just in case that pattern B is
	finally not matched. Consider the two patterns \pattern{return} and
	\pattern{$[$a-z\_$]$+":"}. When a character stream "return" came in, 
	the first pattern matched, but the second is still an option. So,
	if a string "return\_label:" appears one can report that the second 
	pattern matched. But, if the trailing colon is missing, we need to be
	able to go back and report that there was a return statement.
	For this reason, we need two variables:

	\begin{itemize}
	    \item {\tt last\_acceptance}: storing the state index of the last pattern 
	    	that 'won' by acceptance.
	    \item {\tt last\_acceptance\_input\_position}: storing the position in the
    	       the character stream where the last pattern won the match.	    
	\end{itemize}

    Since one incoming character is potentially compared multiple times until it
    is determined to what follow-up state it triggers, one needs a variable to store
    the character that just arrived: {\tt input}.
    
    For the case, that post-conditions
    appear, variables have to be added that store the place where a particular
    post-condition starts, i.e. the place one has to jump back if the post-condition
    is successful:

    \begin{center}    
	{\tt last\_acceptance\_N\_input\_position}
    \end{center}

    where {\tt N} is the numeric identifier of the pre-condition.
    If pre-conditions are involved one needs to store information wether the pre-conditions
    are fulfilled or not. Thus variables need to be provided to store the pre-condition
    results:    

    \begin{center}{\tt pre\_condition\_M\_fulfilled\_f}
    \end{center}
    
    where {\tt M} is the index of the pre-condition.
    The lexical analyser core, generated by \quex produces a single function, let it be
    called {\tt analyse\_this()} that is called in order to initiate the lexical
    analysis: 
    
    
\begin{lstlisting}
QUEX_ANALYSER_RETURN_TYPE
analyse_this(QUEX_ANALYSER_FUNC_ARGS) {
    // (1) variable definitions ___________________________________________________________________________
    //
    //  -- basic required variables
    int                        last_acceptance = -1;
    QUEX_STREAM_POSITION_TYPE  last_acceptance_input_position = (QUEX_STREAM_POSITION_TYPE)(0x00);
    QUEX_CHARACTER_TYPE        input = (QUEX_CHARACTER_TYPE)(0x00);\n

    //  -- variables to deal with post-conditioned patterns (optional)
    //    QUEX_STREAM_POSITION_TYPE  last_acceptance_i_input_position = (QUEX_STREAM_POSITION_TYPE)(0x00);
    //    QUEX_STREAM_POSITION_TYPE  last_acceptance_k_input_position = (QUEX_STREAM_POSITION_TYPE)(0x00);
    //    QUEX_STREAM_POSITION_TYPE  last_acceptance_l_input_position = (QUEX_STREAM_POSITION_TYPE)(0x00);
    // ...	

    //  -- variables to deal with pre-conditions (optional)	
    //    int   pre_condition_i_fulfilled_f = 0;
    //    int   pre_condition_i_fulfilled_f = 0;
    //    int   pre_condition_i_fulfilled_f = 0;
    // ...	

	
    // (2) state transition code  _________________________________________________________________________
    // ...
    	
	
    // (3) pattern action code / terminal states __________________________________________________________
    // 	   (possible function return from here)
    // ...	
}    
\end{lstlisting}    

The way the function is defined leaves opportunities to adapt it to different
types of environments, depending on the definitions of the {\tt
    QUEX\_$\ldots$}-macros.  The following to sections discuss the construction
    of state transition code and code for the terminal states. Then it has to
    be discussed how inverse state machines for pre-conditions are translated
    into code and fit into the picture. A final section explains how to modify
    the analyser function by the definition of the {\tt QUEX\_$\ldots$}-macros.

    
