The {\quex} program is provided in source code with an LPGL license in order
to support understanding the process of lexical analysis in its practical
application. The source code may be used for teaching or to enhance 
the functionality or performance of the software. In order to allow
programmers to get into {\quex}' source code programming, the following sections
describe the way that {\quex} generates lexical analysers step by step. 
Clearly, this chapter is a hacker's guide to {\quex}. Users interested
only in the application of {\quex} may skip this chapter without hesitating.

The generation of the core engine is separate from the handling of modes and
their relationships. In the following sections, only the generation of the core
engine, i.e. a lexical analyser for a single mode, is explained. Each mode
consists of such an engine. The aggregation of engines for each mode to one
single analyser engine is trivial and not subject of further explanation\footnote{The
switching between modes is not much more than the bending of a function pointer to
the mode's analyzer function.}. For further support to get into programming, the
unit tests for each sub-module in the source code may be considered. They can
be found in the {\tt ./TEST/} sub-directory of each module. 

Note, that the following operations are only applied if the core engine is not
to be generated by the {\tt flex} program.  If the core engine is to be
produced by flex then {\quex} only translates the modes into source code readable
by {\tt flex}, receives its output, and adapts some headers so that the {\tt
flex}-produced engine fits into the {\quex}'s framework. The engine produced
by {\quex}, though, is a directly coded engine, rather than a table driven
approach. It is more suited for the handling of unicode and provides
advantages in terms of system memory usage and calculation speed.
