Additionally to the {\tt on\_entry} and {\tt on\_exit} event there is an event
for the {\tt first non-whitespace character} that appears in a line: {\tt
  on\_indentation}. In order to trigger this event a dedicated process keeps
track of newlines and whitespace that follows it. This process happens
virtually in paralell to pattern matching. It knows to states:

\begin{description}
\item[State 1]: Newline has occured and since then, there was only whitespace.
\item[State 2]: There was at least one none-whitespace since newline.
\end{description}
  
At the beginning of lexical analysis, or at the beginning of a line the
lexical analyser is in state 1. When the first non-whitespace character
appears in a line it enters state 2. The transition from state 1 to state 2,
i.e. the appearance of the first non-whitespace character in a line triggers
the indentation event and the correspondent event handler is called. The
transition from state 2 to state 1 and state 1 to state 2 can also appear
inside one single matched pattern\footnote{Imagine, for example a pattern 
$[$ $\backslash${\tt n}$]+$ that matches newlines and whitespace.}.

This event handler is called right before the pattern action is executed that
belongs to the matched pattern. Inside the event handler, in addition to the
{\tt self} reference, the following variable is available:

\begin{description}
\item[\tt const int Indentation]: The distance from the beginning of the line
  to the first non-whitespace.
\end{description}

Note, that the indentation event can be disabled, but only for once. Using the
member function

\begin{lstlisting}
  void  disable_next_indentation_event();
\end{lstlisting}

disables the indentation event for the next time. However, after the next
prevented indentation handling it is enabled again. This comes handy if one
needs to have a 'line-prolonger', such as a Backslash in python and many shell
script languages, or the underscore in VisualBasic. The following mode action
pair would prevent the subsequent line to be considered for indentation
handling:

\begin{lstlisting}
mode SOMETHING {
   ...
   {P_BACKSLASH_FOLLOWED_BY_NEWLINE} {
        self.disable_next_indentation_event();
   }
   ...
}
\end{lstlisting}

If the subsequent line, though, does not end with a backslash the event
handler is automatically active and indentation handling is in place.


