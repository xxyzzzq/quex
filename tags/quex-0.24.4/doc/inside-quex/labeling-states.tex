The previous sections discussed how to generate a state machine for one
isolated regular expression using the Thomson Construction.  The final lexical
analyzer, though, needs to have all possible patterns present, i.e. combined
into an automaton in order to judge which pattern matches from the current
input position. The approach that was chosen in {\quex} was to produce a single
deterministic automaton from the set of state machines for each pattern.  This
process is described in the following sections.

It was already discussed that information about acceptance
\myref{sec:basics/pattern-matching}, storing of input
positions\myref{sec:inside-quex/intro/post-conditions} and dependence on
pre-conditions \myref{sec:inside-quex/intro/pre-conditions} need to be stored
as hidden state variables\cite{} inside the states of the isolated pattern
state machines. Those informations need to be maintained when the states
dissolve into the 'super state machine' that contains all patterns. This happens
by labeling them with origin information.

As for acceptance, no special information carrier is required since the Thomson
Construction and the Hopcroft Optimization treat them appropriately. A 
state in the 'super state machine' has acceptance if one of its original states had
acceptance and it does not have acceptance if none of its original states had
acceptance. It remains to store information about which pattern has raised
the acceptance in a particular acceptance state, i.e. some type of pattern-id.

Without post-conditions an acceptance state stands for storing the input position.
If the pattern that raised acceptance at this point wins, then the analyzer's focus
must be set back to this position and the next analysis starts from this point. With
post-conditions acceptance and storing the input position are no longer the same.
When the core pattern has ended, the input position must be saved. But, at this
point the state machine has not yet reached acceptance. It must first reach the end
of the post condition. Then the input position can be put back to the end of the
core pattern. Thus a flag is required telling wether or not it is required to store
the input position.

The input positions for different post-conditions must be stored in different variables,
since only at the end of the match it can be decided which pattern  actually 
succeeded. A flag indicating that the original pattern was post-conditioned 
tells the code generator to use the pattern-id as indicator for the variable
storing the input positions.

The existence of pre-conditions imply the check at an acceptance state
if a particular pre-condition has been met. Thus a variable is required
containing the pre-condition which has to be met in order to trigger acceptance.
A special pre-condition is the begin-of-line pre-condition. It is a trivial 
pre-condition, since it does not involve an inverse state machine. Thus
it is treated as a separate flag.

Inside {\Quex}, there is a data structure dedicated to store
this information, called class {\tt StateOriginInfo}. It contains
the following fields:

\begin{description}

\item [\tt state\_machine\_id] carries the id of the original state machine. 
      This identifies the pattern to which the state belongs.

\item [\tt state\_index] contains the index of the original state.

\item [\tt store\_input\_position\_f()] gives information about the input
      position to be stored or not. In the table below referenced as {\bf S}.

\item [\tt post\_conditioned\_acceptance\_f()] indicates whether the origin of the 
      state relates to a post condition. In the table below referenced as {\bf post}.

\item [\tt pre\_condition\_id] indicates the id of the pre-condition that has to be fulfilled
      so that the acceptance of the state is valid. In the table below referenced as {\bf pre}.

\item [\tt trivial\_precondition\_begin\_of\_line\_f] indicates wether the pattern
      is supposed to start at the beginning of a line, i.e. after newline or the
      start of the buffer. In the table below referenced as {\bf bol}.
\end{description}

Note, that there is nothing like a flag to indicate 'trivial post-condition end of line'.
This is for the simple reason that this 'trivial' post condition is internally translated
into a post-condition 'end of line or end of file'. There is no mechanism that could
gain some speed here. For the begin of line pre-condition though, one can rely on the
last character that was considered. 

A state in the 'super state machine' contains a list of origins, each origin
corresponds to an original state of the isolated patterns.  The following table
shows the possible combinations of these parameters together with the
acceptance flag {\bf AccF} of the state that may be labeled with these origins:

\begin{center}
\begin{tabular}{c|c|c|c|c|c|}
{\bf Original State}          & \multicolumn{5}{c|}{State containing the Original State}  \\
	                      &            & \multicolumn{4}{c|}{One of its Origins}      \\ \hline
                              & {\bf AccF} & {\bf S} & {\bf post} & {\bf pre} & {\bf bof} \\ \hline \hline
%%			      
normal acceptance state       & must       & yes     & ---        & ---       & --- \\ \hline
normal non-acceptance state   &  ?         & ---     & ---        & ---       & --- \\ \hline
end of post-conditioned core  &  ?         & yes     & yes        & ---       & --- \\ 
pattern                       &            &         &            &           &     \\ \hline
end of post-condition         & must       & ---     & yes        & ---       & --- \\ \hline
acceptance of pre-conditioned & must       & yes     & ---        & yes       & no! \\
pattern                       &            &         &            &           &     \\ \hline
acceptance of begin-of-line   & must       & yes     & ---        & no!       & --- \\
pre-conditioned pattern       &            &         &            &           &     \\ \hline
\end{tabular}
\end{center}

Note, that 'must' for the acceptance flag means that the state in the final
state machine that contains this origin {\it has to be} an acceptance state.
The '?' means that in the original state machine the state was not an
acceptance state, thus the state that contains it together with others may or
may not be an acceptance state. Note also, that pre-conditioned patterns
and trivially pre-conditioned patterns are mutually exclusive. 

In order to help to get easy access to the origin's state type the member
function '{\tt type}' returns direct information about the type of the original
state, i.e. if it was a normal acceptance, a non-acceptance, a post-conditioned
state (end of core pattern), an end of a post condition or a pre-conditioned
acceptance state. In case the constitution does not lead to any valid
interpretation, the function replies with an error. This would indicate that
some functions do not work propperly. This allows for a double-check that
all origins have passed the state machine construction propperly.
    


