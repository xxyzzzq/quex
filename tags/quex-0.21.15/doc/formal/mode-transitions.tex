There are exactly two events that are related to mode transitions: entering
and exiting. Accordingly, one can define inside a mode the two handlers {\tt
on\_entry} and {\tt on\_exit}. In addition to the {\tt self} reference, the
following variables are available:

\begin{description}
\item[\tt const quex\_mode\& FromMode:] Only availabe in {\tt on\_entry} where
  this is the mode {\it from} which the current mode is entered.

\item[\tt const quex\_mode\& ToMode:] Only availabe in {\tt on\_exit} where
  this is the mode {\it to} which the lexical analyser transits.
\end{description}

The {\tt on\_exit} function of the mode that is left is always called before
the {\tt on\_entry} function of the mode that is entered. Note, that it does
not make sense inside transitions to use the {\tt RETURN} statement since at
this point the token-queue is not to be synchronized with a user's token
request. Examples for enter and exit handlers have been given in section
\ref{sec:practical/pattern-action-pairs}.





