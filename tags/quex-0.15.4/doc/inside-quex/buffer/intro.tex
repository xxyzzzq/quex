\framebox{\vbox{NOTE: The information here is not complete, since the buffer 
    handling was re-designed. Add to this chapter inconv support and input
        strategies.}}
\vskip 0.2cm

Instead of relying on direct read and write operations from storage devices and
transmission lines, it is generally advantegous to profit from the fact that
data is usually transported faster when it is transported in some larger
blocks.  When doing lexical analysis one treats the data stream character by
character. To send a 'request' for each character and wait for the reply each
time a new character is read is obviously not a very promissing approach.
Independent on what device or line there is in the background the solution that
comes handy is: {\it a buffer in the system memory}. A buffer loads a larger
blocks of data and keeps them ready-to-read until its content is read completely. Then
it loads the next larger block of data.

\showpic
{figures/basic-buffer.pdf}
{Structure of {\quex}'s buffer.}
{fig:basic-buffer}

The structure of {\quex}s buffer is shown in figure \ref{fig:basic-buffer}. The buffer itself
is not more than a continous chunk of memory. It has two places to store
border characters. One is located at the front to store 'begin of buffer' or 'begin of
file'. The other is located at the end to store 'end of buffer' or 'end of file'.
However, the 'end of file' character may come anywhere from the start of the
buffer to the end. Details about these limit characters are explained below.
Thus the actual buffer size is two elements greater than the actual number
of content elements. A 'current' pointer iterates from left to right\footnote{In this figure
it is 'left-to-right'. This basically means from low memory addresses to 
higher memory addresses. It had absolutely nothing to to with the directionality
of the character encoding.} to go forward. The element to which it points designates
the current input to the lexical analyser state machine. If it reaches a border
new content has to be loaded. The fallback area allows to prevent immediate
loads in the opposite directions if one requests to go back immediately
after loading.

The following sections explain the mechanisms of {\quex}s buffer, some
basic API functions for lexical analysis, the loading procedures, and the
buffer creation procedure.


