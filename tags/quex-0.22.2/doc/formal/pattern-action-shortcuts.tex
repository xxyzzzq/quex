In practical applications for programming languages, the large majority of
patterns to be identified are simple and do not require large algorithms to
handle them. Keywords, such as '`if`', '`then`', and '`foreach`',
operators, such as '$+$' and '`*`' and delimiters such as '`;`', `'
  (`' and '`)`' simply need to send a token telling that such and such
appeared. Their pattern action pairs all look similar to the following:

\begin{lstlisting}
  ...
  {P_KEYWORD_IF} {
       self.send(TKN_KEYWORD_IF);
       RETURN;
  }
  {P_KEYWORD_THEN} {
       self.send(TKN_KEYWORD_THEN);
       RETURN;
  }
  {P_OPERATOR_PLUS} {
       self.send(TKN_OP_PLUS);
       RETURN;
  }
  ...
\end{lstlisting}

Quex allows to write such simple token sendings in a more elegant manner.
If a pattern is followed by a `=$>$` operator and a name _xyz_ then
this is translated into 'pattern sends token with name _xyz_'. The code as
shown above will be internally generated. The above code segment is equivalent
to the following:

\begin{lstlisting}
  ...
  {P_KEYWORD_IF}    => TKN_KEYWORD_IF;
  {P_KEYWORD_THEN}  => TKN_KEYWORD_THEN;
  {P_OPERATOR_PLUS} => TKN_OP_PLUS;
  ...
\end{lstlisting}

For simple patterns that consist of a fixed string and that appear only in one
place it does not make sense to define a pattern explicitly. They can also be
given directly, i.e. one can write

\begin{lstlisting}
  ...
  "if"    => TKN_KEYWORD_IF;
  "then"  => TKN_KEYWORD_THEN;
  "+"     => TKN_OP_PLUS;
  ...
\end{lstlisting}

thus avoiding long lists of pattern definitions external to the mode
definition. Sometimes, though, things are little more complicated.  It might
be necessary to extract some of the information in the lexeme and store it
into the token as in the following pattern action pairs:

\begin{lstlisting}
  ...
  {P_NUMBER} {
    self.send(TKN_NUMBER, atoi(Lexeme));
    RETURN; 
  }
  {P_IDENTIFIER} { 
    self.send(TKN_IDENTIFIER, Lexeme); 
    RETURN; 
  }
  ...
\end{lstlisting}

In these cases, one passes arguments to the send function other then the
token-id. In order to specify additional arguments to the send functions the
token-ids have to be followed by a pair of brackets containing the list of
arguments (possibly separated by comma). This way, the above two pattern
action pairs can be rewritten elegantly as:

\begin{lstlisting}
  ...
  {P_NUMBER}     => TKN_NUMBER(atoi(Lexeme));
  {P_IDENTIFIER} => TKN_IDENTIFIER(Lexeme); 
  ...
\end{lstlisting}

These abbreviations not only speed up the generation of a lexical analyser
tremendously, but they make the lexical analyser descriptions also very
readable. Long lists of pattern-action pairs can be understood and
analysed in a few seconds.

Note, that if an identifier is used that is not declared in a `token`-section, 
    then quex implicitly declares this token, but still a warning is reported.
In order to get things setup quickly, it is possible to complete renounce on
{\tt token} sections. However, for a good style the output of these warnings
might be pasted into one of the {\tt token}-sections.
