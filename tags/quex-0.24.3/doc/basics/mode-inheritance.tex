Object oriented programming introduced the concept of {\it
inheritance}\cite{}, in the sense that two classes can have a 'is-a'
relationship. The class of 'cars' may for example be related to the class of
'vehicles' by an 'is-a' relationship. This means that 'cars' inherit all
attributes of the broader class 'vehicles' and may add some specific
attributes to it. Such inheritance relationships are directional. In our
example this means that every object that belongs to the class of cars also
belongs to the class of vehicles (every car is a vehicle). However, not every
object that belongs to the class of vehicles (e.g. a submarine) does belong to
the the class of cars. Mode inheritance in {\quex} works similar and effects
the following three mechanisms:

\begin{itemize}
\item {\it Patterns}: If a mode B is derived\footnote{The statements
    'B is derived from A', 'A is base class of B', and 'B inherits from A'
    are equivalent in meaning in the context of object oriented programming. 
    In the same way the term 'derivation', 'base class', and 'inheritance'
    shall express mode relations in mode oriented lexical analysis.}
  from a mode A, then it overtakes all of its patterns and the correspondent
  pattern match actions. This is practical if one wants to ensure that 
  a certain mode does react on certain patterns. In this case it has to be
    derived from the mode that shows the desired behavior. 
    
    {\bf Rule}: If mode B is derived from A and both have a common pattern
    {\sf XYZ} where mode B relates it to a pattern action $XYZ_B$ and mode A
    relates it to a pattern action $XYZ_A$, then the pattern action $XYZ_B$ is
    canceled. {\it A base pattern-action pair overrides any derived pattern
    action pair of the same pattern.} 

  
\item{\it Event Handlers}. If a mode B is derived from a mode A, then B's
  event handler will be executed after A's event handlers. For example, A may
  have an event handler '{\tt on\_entry}' for entering mode A. Mode B itself
  might have an event handler for '{\tt on\_entry}'. If the lexical analyser
  now enters mode B, the event handler of its base mode A is first executed
  and then the event handler of class B itself.
  
  {\bf Rule:} {\it Event handlers are executed in sequence from base to
    derived classes.}
  
\item{\it Mode Access}: If a mode B is derived from a mode A, then 
  any permission of entry or exit to A is also a permission for B to
  enter or exit. The philosophy behind this is that that due to pattern
  inheritance mode B will behave like mode A, so it can be trusted the same
  way that one trusts A. 

  {\bf Rule:} {\it Entrys and exits that are allowed for a base class
    are also allowed for all its derived classes.}
  
\end{itemize}

Figure \ref{fig:mode-inheritance} shows an example where a mode A acts as a
base mode for mode B. Internally, {\quex} creates a new mode B* that contains
the patterns that were explicitly givent by the definition of B and the
patterns that are inherited from A. The pattern match action of {\sf XYZ} from
mode B is omitted, since mode B has to behave like mode A for patterns
specified in mode A. The {\tt on\_entry} and {\tt on\_exit} functions are of
mode A are pasted into the own event handlers so that they are executed before
them. What is not displayed in the figure is access inheritance. When {\Quex}
checks the entry and exits of a mode it consideres automatically also the
base classes. If there is no base mode that acts like a door opener to a
particular mode, then transits will cause a run-time error.

\showpic
{figures/mode-inheritance.png}
{Mode Inheritance.}
{fig:mode-inheritance}

In conclusion, it can be said that inheritance is a very useful tool to
implement some 'standard' behavior in a base mode. Derived modes can then
comply to this standard simply by being derived from this mode. Also, for the
control of mode transitions the is-a relationships of inheritance provides
much flexibility. Mode inheritance allows to think about mode relationships
very naturally and helps to avoid code duplication.
