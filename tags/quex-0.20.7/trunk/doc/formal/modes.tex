{\Quex} allows to specify characteristics of modes concerning 
transitions and inheritance. The keyword {\tt mode} starts the
definition of a lexical analyser mode. Then it can be directly followed by a
curly bracket that starts the pattern-action pair definitions as shown in
figure \ref{fig:formal-mode-definition}a. This is the most simple way to
define a mode.

\begin{figure}[htbp] \leavevmode
      a)
      \begin{center} 
        \begin{lstlisting}
   mode MY_MODE {
     ...
     // pattern-action pairs of MY_MODE
     ...
   }          
        \end{lstlisting}
      \end{center} 

      b)
    \begin{center}
        \begin{lstlisting}
   mode MY_MODE :
      END_OF_FILE
      WHITESPACE_EATER
      DOCUMENTATION_TRIGGER
      {
     ...
     // pattern-action pairs of MY_MODE
     ...
   }          
        \end{lstlisting}
    \end{center} 

      c)
    \begin{center}
        \begin{lstlisting}
   mode MY_MODE :
      END_OF_FILE
      <inheritable: yes>
      <exit:        OTHER_MODE>      
      <restrict:    exit>
      {
     ...
     // pattern-action pairs of MY_MODE
     ...
   }          
        \end{lstlisting}
    \end{center} 
  \caption{Definition of mode characteristics.} 
  \label{fig:formal-mode-definition}
\end{figure}

In order to define options or base modes the name of the mode has to be
followed by a colon as in the examples in figure
\ref{fig:formal-mode-definition}b and \ref{fig:formal-mode-definition}c.
Note, that if a mode specifies base modes and options, the options are
specified after the base modes. 

Base modes directly follow the ':' sign. The list of base modes is whitespace
separated. When the first $<$ sign arrives, {\quex} starts reading the
options. Options are bracketed by $<$ and $>$ signs and multiple options 
are also listed whitespace separatedly. The following options can be
specified:

\begin{itemize}
\item {\tt $<$inheritable: $[$ "yes", "no", "only" $]$ $>$$]$} restricts the
  inheritance relationships. If {\tt yes} is specified, it is allowed for any
  other mode to be derived from this mode. If {\tt no} is specified and
  another mode tries to derive from this mode, an error will be printed.
  Specifying {\tt only} tells that this mode cannot be a lexical analyser mode.
  Its reason for existance is to support derived modes. This is similar to the
  concept of abstract classes in C++ or interfaces in Java.
  
\item {\tt $<$entry: {\it mode-name} $>$} tells that it is allowed that the
  lexical analyser enters this mode from mode {\it mode-name} or one of its
  derived modes. If no entry mode is specified, then all modes are allowed.
  
\item {\tt $<$exit: {\it mode-name} $>$} tells that it is allowed that the
  lexical analyser exits from this mode to mode {\it mode-name} or one of its
  derived modes. If no exit mode is specified, then all modes are allowed.
  
\item {\tt $<$restrict: $[$ "entry", "exit" $]$ $>$} turns of the
  admissibility of derived modes for entry or exit.  If {\tt entry} is
  specified the entry is restricted, so any other mode than the the listed
  entry modes is forbidden even if it is derived from one of them.
  Respectively, specifying {\tt exit} restricts the list of exit modes.
\end{itemize}

Options are optional, i.e. they do not need to be specified. But, they
facilitate the debugging and add confidence that the mode transitions are 
safe and sound. The list of options is then to be followed by an opening
curly bracket '$\{$' that opens the list of pattern-action pairs. These are
discussed in the following section.




%%% Local Variables: 
%%% mode: latex
%%% TeX-master: t
%%% End: 
