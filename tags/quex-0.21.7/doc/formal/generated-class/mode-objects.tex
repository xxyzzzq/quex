Modes themselves are implemented as objects of classes which are derived from
the base class {\tt quex\_mode}.  Those mode objects have member functions
that provide information about the modes and possible transitions:

\begin{lstlisting}
        bool  has_base(const quex_mode& Mode,      
              bool PrintErrorMsgF = false) const; 
        bool  has_entry_from(const quex_mode& Mode,
              bool PrintErrorMsgF = false) const;  
        bool  has_exit_to(const quex_mode& Mode, 
              bool PrintErrorMsgF = false) const;
        const int     ID; \\
        const string  Name; \\
\end{lstlisting}

The first three member functions allow to get information about the relation
to other modes. If the flag {\tt PringErrorMsgF} is set than the function will
print an error message to the standard error output in case that the condition
is not matched. This comes very handy when using these functions in {\tt assert}s or
during debugging. The functions can be applied on a given mode object or
inside the {\tt on\_entry} and {\tt on\_exit} functions with the {\tt this}
pointer.  In a pattern action pair, for example, one might write

\begin{lstlisting}
     if( PROGRAM.has_base(self.mode()) )
         cerr << "mode not a base of PROGRAM: " << self.mode_name() << endl;      
\end{lstlisting}

For the end-user these functions are not really relevant, since {\quex}
itself introduces {\tt assert}s on mode transitions and provides convienient member
functions in the lexical analyser class to access information about the
current mode.
