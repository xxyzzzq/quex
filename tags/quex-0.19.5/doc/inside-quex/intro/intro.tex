The following sections elaborate on the process of generating code for a directly coded lexical
analyser. Each lexical analyzer mode consists of such an independent
code fragment embedded into a function. In a first section, the big picture is
discussed. This gives an overview about the major issues involved and build a
basis for later sections to refer to. Many books discuss in detail the
Thomson-construction and the Hopcroft-Optimization, but few (the author does
	not know a single one) discusses the construction down to
code-generation and its specialities in detail. {\Quex} was created to handle
efficiently Unicode character sets. As a consequence the table-driven approach,
	    such as in lex \cite{} is no longer suitable%%
%%  
\footnote{In table driven analysers the table size is the square of the number
    of characters. Unicode provides currently \$10.FFFF code points. A table
	for this character set would blow the memory of todays computers (The
		author writes these lines at the beginning of the 21$^{st}$
		century).  Compressing the table might slow down the speed of
	    the lexical analyser significantly.}.
%% 
Thus, {\quex} implements a {\it directly coded lexical analyser} where no state
variable is needed and no character-trigger table. It is also a much faster
approach, since no memory has to be referenced%%
%%
\footnote{Note, that an expression such as '{\tt vector$[$i$]$}' implies
    generally a multiplication ({\tt offset = i * sizeof(type)}), an addition
	({\tt adress = vector + offset}), and a memory referencing  operation
	({\tt object = *adress}).}.
%%	      

