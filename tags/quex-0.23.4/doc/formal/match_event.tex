{\Quex} provides a special event is executed at the moment any
match is executed. This allows the user to do some things---for 
example for debugging  purposes--- at any pattern match. The
event handler name to do this is {\tt on\_match}. The code
fragment that follows this name is considered to be the action
which is executed before any match action is executed. Every
mode has its own {\tt on\_match} event handler. However,
     these event handlers can be inherited in the same way
     as {\tt on\_exit} and {\tt on\_entry} are inherited.


The event handler receives the two arguments '{\tt Lexeme}', the pointer 
to the first character of the lexeme and
'{\tt LexemeL}', i.e. the length of the lexeme that was matched. The following shows
an example, where the user wants to do some statistics 
on the number of matches, the number of backslashes and the 
total length that was matched in the mode '{\tt CORE}'.

\begin{lstlisting}
mode CORE :
...
{
   ...
   on_match {
       match_count        += 1;
       backslashe_n       += __count_backslashes_in_string(Lexeme);
       total_match_length += LexemeL;
   }
   ...
} 
\end{lstlisting}
