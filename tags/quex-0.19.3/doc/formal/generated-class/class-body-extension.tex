Instead of deriving a class from the lexical analyser, it is sometimes more
convenient to simply paste some more content into the class body. This can be
done using the keyword {\tt body} outside the mode definitions. The following
code, for example, pastes the definition of an integer vector into the class'
body, so that it can later be used to keep track of indentation columns.

\begin{lstlisting}
body {
  std::vector<int>    indentation_stack;
}
\end{lstlisting}

Note, that the namespace prefix {\tt std::} is used. This is because the class
definition's body appears inside a header file and in header files one better
uses absolute namings, rather than abbreviating using namespaces. The newly
defined member can then be used inside any pattern action or event handler via
access of the {\tt self} reference, e.g.

\begin{lstlisting}
mode SOMETHING {
   ...
   on_indentation {
      ...
      // close any block that has higher indentation
      while( Indentation < self.indentation_stack.back() ) {
          // send token that indicates: block-end
          self.send(TOKEN_END_OF_BLOCK);
          // cut indentation block from list
          self.indentation_stack.pop_back();
      }
      
   }
   ...
}
\end{lstlisting}

