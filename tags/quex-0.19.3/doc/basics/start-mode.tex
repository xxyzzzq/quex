In most practical application it is not a good idea to start in a void initial
mode, so {\quex} requires an initial mode explicitly to be specified. This
happens using the {\tt start} keyword, i.e. one has to type somewhere in
between the mode definitions (but best at the very beginning):

\begin{lstlisting}
  start = PROGRAM
\end{lstlisting}

Let the mode-stubs be stored in a file called {\tt simple.qx}. The previous
sections defined patterns, token-ids, and modes. Together with these
definitions one can now start with the specification of pattern-action pairs
as discussed in the following section.
