The last chapter provided a functioning example for the usage of {\quex}.  At
this point the user should have a clear idea about the basic concepts and how
they are applied. This chapter discusses the details of usage and minor
features that may not have been mentioned in the previous chapters. The first
sections discuss the specification of patterns, mode characteristics,
pattern-action pairs, and mode transitions. Section
\ref{sec:formal/generated-class} discusses the lexical analyser class that
{\quex} creates with all members and member functions that the user might want
to use.

In some cases, the provided functionality of the generated lexical analyser
class might not be sufficient, so section \ref{sec:formal/derivation}
discusses the formalities to implement a derived class and its usage. The
generated code contains some macro switches that allow to control the behavior of
the analyser at compile time. Section \ref{sec:formal/macros} discusses the
usage of those macros. Finally, section \ref{sec:formal/command-line-options}
explains all command line arguments that can be passed to {\quex}.

