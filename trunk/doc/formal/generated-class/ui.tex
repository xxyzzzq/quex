Before the lexical analyser can be used, one has to create an instance of it.
Assuming that '{\tt the\_lexer}' is the name of the lexical analyser engine
(see command line option {\tt --engine}) the following two constructors are
provided:

\begin{description}
  \item[\tt the\_lexer(const istream* in\_stream, ostream* out\_stream = 0);]
  \item[\tt the\_lexer(const string\& in\_filename, ostream* out\_stream = 0);]
\end{description}

The first constructor accepts an input stream of any kind and an optional
output stream for unmatched characters. The second constructor directly
accepts a filename instead of a stream. The stream or the filename act
as the 'root-file' for the lexical analysis. The lexical analyser is now
ready to run. Tokens are received by calling the member function

\begin{description}
  \item [\tt token::id\_type   get\_token(token* result\_p);]
\end{description}

It initiates a lexical analyser process and fills the object at {\tt
  result\_p} with the next token. Internally, tokens may be stacked 
and not every call to {\tt get\_token()} initiates an analysis. But the
user does not care. He simply receives a sequence of tokens through this
function until a token arrives with the the token-id 

\begin{center}
\begin{verbatim}
 quex::token::ID_TERMINATION
\end{verbatim}
\end{center}

This id is by default defined as zero, because also many parsers require that.
This id tells that the input has been totally treated and no further token
will arrive. In order to keep track of different versions of generated
lexers the member function

\begin{description}
  \item [\tt const char* version() const;]
\end{description}

returns a string telling about the version (see also command line option {\tt
  --version}), and the date when {\quex} produced the engine.  The current
line and column numbers can be accessed using the functions

\begin{description}
  \item [\tt int  line\_number() const;]
  \item [\tt int  line\_index() const;]
  \item [\tt int  column\_number() const;]
  \item [\tt int  column\_index() const;]
\end{description}

where the '*\_number()' functions return the information where the last
pattern started, and the '*\_index()' functions return the information where
the last pattern ended.



