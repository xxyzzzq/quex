The {\tt accumulator} is a member that allows to stock strings to communicate
them between pattern-actions. In the practical example in section
\ref{sec:practical/intro} the string contained in string delimiter marks is
accumulated until the {\tt on\_exit} handler was activated, i.e. the {\tt
    STRING\_READER} mode is left. Insider the handler, the string is flushed
    into a token with a specific id {\tt TKN\_STRING}.  The accumulator
    provides the following functions:

\begin{lstlisting}
  void   add(const char*);                  
  void   add(const char);                    
  void   flush(const token::id\_type TokenID);
  void   clear(); 
\end{lstlisting}

The {\tt add}-functions add a string or a character to the accumulated string,
the {\tt flush()} function sends a token with the accumulated string and the
specified token-id. Finally, the {\tt clear()} function clears the accumulated
string without sending any token.

