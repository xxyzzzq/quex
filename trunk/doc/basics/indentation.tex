With the rise of the Python programming language the use of indentation as
scope delimiter has become popular. Indeed, it maybe the most efficient method
to delimit scope with the least amount of additional characters
\cite{Safer2004}, such as '\{' and '\}' in C-styled languages.  {\Quex}
provides a convinient mechanism to handle indentations which runs quasi in
paralell to the pattern matching: {\it indentation events}.  

Figure \ref{fig:indentation-principle} displays in an example the principle of
indentation events. Whenever the lexical analyser reaches the first
non-whitespace in a line, an indentation event (indicated as a little star in
the figure) is triggered. The lexical analyser engine then calls a user
defined indentation handler. The numbers at the indentation events indicate
the number of characters that the indentation spans. They are passed to the
user's indentation handler as arguments. The indentation handler, then, can
then keep track of indentation blocks or whatever his little heart desires.


\showpic
{figures/indentation}
{Principle of Indentation Events}
{fig:indentation-principle}

Note, that it is not trivial to express indentation in terms of pattern action
pairs solely based on regular expressions. It is not enough to define a
pattern such as 

\begin{quote}
  \begin{verbatim}
          P_INDENTATION_CATCHER    "\n"[ ]*
  \end{verbatim}
\end{quote}

That is a newline followed by whitespace. Imagine, one introduces a comment
sign such as the traditional '\#' for 'comment until newline'. The comment
eating pattern would be at first glance:

\begin{quote}
  \begin{verbatim}
          P_COMMENT_EATER    "#"[^\n]*\n
  \end{verbatim}
\end{quote}

That is a '\#' followed by anything but newline and then one newline. The
action related to this pattern would have to put pack the last newline.
Otherwise the indentation catcher which starts with a newline can never
trigger. In this particular case, this problem can be solved by deleting the
last newline from the comment eater pattern, knowing that after 'as many
not-newline as possible' there must be a newline, i.e.

\begin{quote}
  \begin{verbatim}
          P_COMMENT_EATER    "#"[^\n]*
  \end{verbatim}
\end{quote}

The last newline is then eaten by the indentation catcher. However, the main
problem remains: with such a design {\it patterns need to know about each
  other} to work propperly! In an environment of many different modes which
are additionally related by inheritance, it is difficult to guarantee that all
patterns 'respect' the quirks of each other. They better work independently.
The indentation events of {\quex} provide a means that allows one to avoid
inter-pattern dependencies.

Similarly, catching indentation with 'pre-condition newline plus whitespace',
i.e.  {\tt \verb|^[ \t]*|} is fragile, in a sense that another pattern that
contains newline plus whitespace might hinder this pattern from triggering.
In a lexical analyzer with dozens or hundreds of patterns this becomes
quickly unmanageable. Errors that arise from patterns defined somewhere
else are very hard to find and require a lot of insight into the actual
process of lexical analysis. Using the '{\tt on\_indentation}' event
handler ends up in a much clearer and safer design. For more information
about the pre-condition newline pitfall see section
\ref{sec:formal/patterns/context-dependent-pitfalls}, page
\pageref{formal/patterns/context-dependent-pitfalls}.

{\bf Caveat:} 

Note, that if a pattern contains more than one newline then
only the indentation event concerning the last newline ist triggered! Imagine
a pattern

\begin{lstlisting}
      STRANGER   " "*"hello"[\n]+" "*"world"[\n]+" "*"how are you?"
\end{lstlisting}

then the following pattern would match 

\begin{lstlisting}
     hello
   world
        how are you?
\end{lstlisting}

If this matches, then the lines of 'hello' and 'world' do not trigger an
indentation event. So, when dealing with indentation based scoping such strange
things are best avoided\footnote{Probably, it is anyway hard to think of a case
    where such pattern definitions are not to be avoided.}.
