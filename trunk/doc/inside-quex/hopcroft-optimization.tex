In some cases, such as in the example shown in figure \ref{fig:hopcroft-1}a,
the structure of a state machine can obviously be optimized.
Note, that the sole task of the state machine is to determine
if the sequence of characters appeared that matches a pattern.
In this sense, there is no difference between all acceptance
states. As a direct consequence, if there is a set of acceptance
states, such as state 1, 2, and 3 in figure \ref{fig:hopcroft-1}a, where all characters 
trigger to another acceptance state, then those acceptance states
can obviously be combined into a single one. This section
concretizes this idea and develops the process known as 
the Hopcroft optimization.

\showdpich
{figures/hopcroft-before.pdf}
{figures/hopcroft-after.pdf}
{Combining equivalent states of a state machine.}
{fig:hopcroft-1}

An essential concept of the Hopcroft Optimization is the concept of {\it
    equivalent states}. State machines are equivalent if all possible following
    sequences of triggers lead to the same result.  Broken down to states, one
    requires for any two states A and B to be equivalent that all character
    streams that cause failure starting A will cause failure starting from B.
    Further, all character streams that cause acceptance starting from A cause
{\it equivalent acceptance} starting from B. Concretely, this means
that for all possible triggers $t$, one requires that the follow-up state
from A on $t$ is in the same state set than from B on $t$.

In the case of a single pattern, the set of states can be split up into 
two state sets that build the original worklist: the set of states that are 
acceptance states and those that are not. These two state sets are the initial
worklist. For each state set X in the worklist one has to check wether it
behaves 'abnormal', wether it triggers on a character $\alpha$ to a different
set of states then the rest. Now, this state set is split into 'normal' states $X_0$
and 'abornal' states $X_1$. The state set X from the worklist is then
replaced by the two state sets $X_0$ and $X_1$. This process can then be 
repeated until the worklist does not change anymore, i.e. no state sets 
have to be split up, because the represent equivalent states.

\showdpich
{figures/hopcroft-2-before.pdf}
{figures/hopcroft-2-after.pdf}
{Combining equivalent states of a state machine with states of different origin.}
{fig:hopcroft-2}

As long as one deals with a single isolated pattern things end up here.  For
isolated patterns, it only matters if a match occured or not. For the combined
state machine, that represents the whole bank of patterns more information is
to be considered. Such a case is depicted in figure \ref{fig:hopcroft-2}. Two
acceptance state that report different patterns as 'winner' are not equivalent.
Thus, the state set of initial equivalent acceptance states needs to be redefined.
Note, that the remaining procedure remains the same -- one only has to discuss
the change of the initial set of equivalent acceptance states for the case that
multiple origins are involved.  Here, the term {\it equivalent acceptance} has
to be revisited. 

\definition{As long as there are no origins involved, then two acceptance states are
    equivalent, if and only if -- well -- they are acceptance states.}

For the combined state machine, though, it is important to know what pattern,
    i.e. original state machine, actually reached acceptance, i.e. matched.  It
    is possible that multiple patterns match at the same time. Thus, one has to
    concentrate on the dominating original pattern (see section
	    \ref{dominating-pattern}, page \pageref{dominating-pattern}).

\definition{If a state machine contains origins, then two acceptance states are equivalent,
       if and only if their dominating original patterns are the same.}

Post conditions do not change anything at this setup, because the acceptance 
state of the post-condition is only reached through the state that represents
acceptance of the core pattern. Pre-conditions, though, add some restrictions.
At this point, it makes sense to imagine the code produced to deal with pre-
conditions:

\begin{lstlisting}
     // entry of a state with some pre-conditions:
     if      PreCondition_Pattern5 == True: acceptance = Pattern5
     else if PreCondition_Pattern2 == True: acceptance = Pattern2
     else if PreCondition_Pattern6 == True: acceptance = Pattern6
     else:                                  acceptance = 7
     // lesser priveledged patterns cannot be dealt with:
     //     if PreCondition_Pattern36 == True: acceptance = Pattern36   // does this 
     //     if PreCondition_Pattern0 == True:  acceptance = Pattern0    // make sense ?
     ...
\end{lstlisting}

Let us assume that the pattern priviledges are ordered the same way as they
appear in the code fragment, i.e. 5 has the highest priority, if not than comes
2, if not then comes 6, and if the precondition of 6 does not hold then comes
the unconditional acceptance of pattern 7. Lesser priviledged patterns
(conditioned or unconditioned) do not matter, since the pattern 7 takes it all.
Lets call the set of origins from the most priviledged one to the first
unconditional state (if it exists) the {\it list of possible
    acceptance origins}. Now, the following can be set about equivalent
    acceptance states:


\definition{If a state machine contains preconditioned origins, then two acceptance states are equivalent,
       if and only if their list of possible origins are the same.}

The only modification one has to do to the Hopcroft algorithm is to adapt
the construction of the initial set of acceptance states. It has to be
split according to the above criterion. The splitting procedure considering the
triggering to equivalent states remains the same. Here is the algorithm 
in pseudo-code:

....

Traditionally, the decision wether a set is to be split is made traditionally \cite{}
based on single characters. This method works fine for ASCII character streams
but becomes uneffective with Unicode (see also page \pageref{}). Again, the author
developed a method that determines equivalence based on character ranges that cuts
down the computation time to a tiny fraction.


       
