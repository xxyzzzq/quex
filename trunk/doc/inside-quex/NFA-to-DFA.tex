The conversion of a non-deterministic finite automaton (NFA) to a deterministic
finite automaton (DFA) brings the advantage that at a particular point in time the state
machine is in one particular state---not in multiple states as with the NFA. This not only reduces
complexity, but also decreases memory requirements (to store state indices and
	pattern match-management) and computation time of the resulting lexical
analyser. The key to do so is to combine states that the NFA takes simultaneously
into a single state. Whenever a state in the NFA triggers on the same  character
to more than one state, the states that are reached by this trigger are combined
into one single state. The process relies on two sub-processes described in this section:
1) construction of a so called epsilon closure, and 2) determining unique state combinations
for the range of all triggers. 

The core idea of this process is to find all possible super states of the NFA, i.e. 
the state combinations that can exist at a particular point in time. Then trigger
sets have to be found that trigger the transition from one state combination to 
the other. The state combinations can then be 'renamed' as a single state and
it is sure that the resulting state machine is deterministic. By requesting
that the acceptance state of a combined state is acceptance as soon as one
of its 'member states' is acceptance the entire mechanism of the NFA is preserved.


\showpic
{figures/epsilon-closure.pdf}
{The $\epsilon$-closure.}
{fig:epsilon-closure}

The notion of $\epsilon$-triggers enabled a very efficient method to
sequentialize, parallelize and repeat state machines. As soon as a state
machine contains an $\epsilon$-trigger it becomes a NFA. If a state A transits
on an $\epsilon$ trigger to a state B, then the state machine is in state A
{\it and} B simultaneously as soon as A is reached, because an
$\epsilon$-trigger does not require any input. This is one reason why the
result of those operations is an NFA. In order to help with the $\epsilon$-transitions
the $\epsilon$-closure is defined:

\definition{An {\bf Epsilon Closure} of a state $n$ is the aggregation of all
    states that can be reached from $n$ via an epsilon transition.}

Consequently, when treating a state, one first collects all states that can
immediately be reached via $\epsilon$-transition into a new state. This is
shown in figure \ref{fig:epsilon-closure}. This new state $n0$ contains all transitions of all
states of the $\epsilon$-closure. It is supposed to be equivalent to the 
super state of the NFA which is at the same time in all of the states of the epsilon closure.
Note, however, even though state 23 in the
figure is part of the $\epsilon$-closure, as a target state it is kept isolated
refrerred with its 'maiden name' rather then as part of $n0$
 %%
 \footnote{From a super state point of view, state 23 which is reached through
     the recursion from itself, is not the same as the state 23 being entered
	 from state 0. Also, $n0$ as being a 'super state ' does not belong to
	 the same 'world' as the original states.}. 

\showpic
{figures/NFA-to-DFA-transitions.pdf}
{Transscription of a state set into a single state of an DFA.}
{fig:NFA-to-DFA}

The next step consist in finding state combinations and the trigger sets by which
they are reached from the new state $n0$. Figure \ref{fig:NFA-to-DFA} shows an example
of a state $n0$ where some characters trigger to multiple states at the same
time. This means, that the state machine would be in more than one state at the
same time, and can, therefore, not be a DFA. On the other hand, one can definitely
say that {\it one particular character} triggers {\it one state combination}. 
Combining the states of each character trigger to a state combination, again allows
the one-to-one (injective) mapping from a trigger to a state---and the conditions
for a DFA are met. The search of a distinct trigger map consist of the following:

\definition{A {\bf distinct trigger map} from a state $n$ is based on target
    state combinations that are reached by exactly the same trigger. A set of
	target states can be combined, if there is at least one common trigger
	(character code point) by which they are reached from state $n$.  When
	all target state combinations are found, they can be 'renamed' as
	single states, and the trigger map describes a one-to-one mapping from
	incoming characters to target states.}

\showpic
{figures/NFA-to-DFA-trigger-history.pdf}
{Combinations of follow-up states displayed against the code points of a character
    set -- similar to a time history}
{fig:NFA-to-DFA-ranges}    

Traditionally \cite{}, \cite{} the detection of suiteable state combinations 
considers each single character of the trigger map isolated.  This works fine, as long as the
code set does not exceed the oldtimer 8-bit ASCII character set. Each state in
the NFA requires then 255 checks.  For a Unicode-based lexical analyser this behaves
significantly different. Unicode 5.0 currently has \$10FFFF, i.e. decimal 1.114.111,
	      code points -- not to speak about code points for private usage.
	      Consequently about 140.000 times checks would have to be performed for each state in the state machine.
	      In order to avoid such a caste in calculation time, the author of {\quex} developed a
	      different algorithm based on character ranges as shown in figure \ref{fig:NFA-to-DFA-ranges}. 
The algorithm identifies state combinations which are related to certain overlapping ranges
of characters. 

As mentioned in \ref{sec:character-range-representation}, the {\quex} engine
relies on a description of triggers in terms of number ranges.  Instead of
checking each character, the algorithm considers the numeric ranges $R_{A,B}$
in which a state $A$ triggers to state $B$. If a range $R_{A,B}$ and $R_{A,B}$
intersect, then $B$ and $C$ are states which are entered simultaneously in the
NFA. Thus, they have to be combined in the DFA. For each set of target state
combination the intersection of all trigger ranges gives the range on what the
combination is triggered. It can now be 'renamed', i.e.  a new state can be
created that is triggered based on any character that fall into the
intersection. 

The computation of intersections between all possible target states can also be
simplified.  Again, the line-up of the ranges on the time line comes handy
(figure \ref{fig:NFA-to-ranges}).  This time line is
implemented by 'begin-trigger-target-state' and 'and-trigger-target-state'
commands that appear at certain code points. Finding intersecting ranges then
consists only in walking down the timeline, adding and removing target states
on any such sommand, and storing the intersecting domains together with the
target state combinations. With this approach it is possible to compute
most suiteable lexical analyzers in the frame of few seconds on present day
computing machines.

Now, that multiple states are combined into one state, the question remains about the
acceptance type of the combined state? The answer becomes simple, though, considering that we try to
match patterns and that a pattern match relates to an acceptance state:

\begin{itemize}
\item If no state in the state combination is an acceptance state, than the 
      state representing the combination {\it is not an} acceptance state.
\item If at least one state in the state combination {\it is an} acceptance state, than 
      the state representing the combination is an acceptance state. This is 
      so, since a pattern that matched through one path is not dismissed
      because another path does not match. For example, the regular
      expression \pattern{if|ignore} matches on "if", i.e. the 'first path'
      as soon as the "f" comes in, even though, the second path expects a "g" and
      would be non-acceptance. 
      Since it is only required that one pattern matches, the acceptance 
      of the first path is sufficient. 
\item If two patterns in a combinatino trigger acceptance, this does not pose
      an exception since there exists preference rules that trigger if two
      patterns match on the same number of characters \myref{}.
\end{itemize}

The transformation from NFA to DFA is a process that is not only applied during
the construction of state machines for isolated patterns, but also on the state
machine that combines all patterns to be matched. In this case, the information
about the origins have all to be kept. This includes the information about the
original state. If a non-acceptance state from pattern X is combined with an
acceptance state from pattern Y, we want to be sure, that the lexical analyser
does not notify that X matched. A correct way to determine the winner pattern
is to consider the origins that were acceptance states, determine the most priviledged
pattern that matches (see section \ref{priviledged-patterns}) and this is the
winner.

The following very important assumption has to hold for the process
of converting a set of state machines, i.e. a NFA, into a single state
machine, i.e. a DFA:

\definition{
Let $P_{a, b}$ be the set of paths from a state $A$ to a state $B$
in a state machine (constituting a parallel branch of the NFA). 
Let $S_{DFA}(A)$ be the set of states that contain $A$ in the resulting
DFA, and let $S_{DFA}(B)$ be the set of states that contain $B$ in the
resulting DFA. For the DFA to uniquely represent the NFA, one must
assume the set of path $P^*(S_{DFA}(A),\,S_{DFA}(B))$ is equal to
the original one, i.e.
%%
\begin{center}
            $P^*(S_{DFA}(A),\,S_{DFA}(B))$ = $P_{a, b}$
\end{center}
}

A direct consequence of this is that if a state X in a pattern can only be
reached over a state Y, then in the resulting state machine this holds equally.
The following algorithm demonstrates in pseudo-code the process of constructing
a DFA that corresponds to an NFA. It can be applied for single patterns, as
well as for the combined state machine that represents the lexical analyser.

\begin{lstlisting}
state-set-worklist = { [ init-state ] }

while worklist not empty:
    new worklist = empty
    for each state-set in worklist:
        ec = epsilon-closure of state-set
	new-worklist = distinct-target-state-combinations of ec
	set triggers for ec

    worklist = new-worklist	

\end{lstlisting}
