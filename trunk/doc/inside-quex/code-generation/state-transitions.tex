The dynamic behavior of a state machine is uniquely determined by the
transitions of its states. For a DFA, it can be assumed that it is in one
distinct state when an incoming character arrives. The incoming character
determines, according to the triggers of the state, what the subsequent state
will be.  The next to sections discuss the code generation for a 
state of the lexical analyser state machine.
The operations to be coded for a state can be subdivided into writing
a state header (section \ref{}) which does some bookkeeping and writing
code for the trigger transitions (section \ref{}).


As mentioned earlier, \quex's engine is based on character ranges. This enables
a fast algorithm to bracket the incoming character. Similar to binary search\cite{},
  it determines if the incoming character lies above a border of the
  middle interval, then higher and lower intervals are checked. Of course, 
  one first checks if the character fits any interval. Imagine an acceptance state
  for pattern 21 that has the following trigger map:

\begin{tabular}{lll}
    mismatch             & $\longrightarrow$  & terminal 21 \\
    $[$A-H$]$, $[$J-Z$]$ & $\longrightarrow$  & state 94 \\
    ':'                  & $\longrightarrow$  & state 95 \\
    I                    & $\longrightarrow$  & state 99 
\end{tabular}

The current implementation produces the following code to implement 
these transition rules:

\begin{lstlisting}
    if( input >= 58 && input < 91 ) {
        if( input < 65 ) {
            if( input < 59 ) {
                goto QUEX_LABEL__ENTRY_95;
            } else {
                goto QUEX_LABEL__TERMINAL_21;
            }
        } else {
            if( input < 73 ) {
                goto QUEX_LABEL__ENTRY_94;  
            } else {
                if( input < 74 ) {
                    goto QUEX_LABEL__ENTRY_99;
                } else {
                    goto QUEX_LABEL__ENTRY_94;
                }
            }
        }
    }
    // no trigger triggered 
    goto QUEX_LABEL__TERMINAL_21;
\end{lstlisting}

The first if-statement checks wether the {\tt input} lies in the borders of the
trigger map. If not, one can directly jump to the terminal state 21. Then one
checks wether the input is less than 65, i.e. an A, and then brackets the input
down for each interval. This way of generating code seems to leave room for
improvement, though.  To check wether the input is ':' (character code 58) we
test against $>=\,58$, $<\,91$, $<\,65$, and $<\,59$, instead of a simple check
$==\,58$. A hand written transition code may look like the following:

\begin{lstlisting}
    if( input == 58 ) goto QUEX_LABEL_ENTRY_95;
    if( input == 73 ) goto QUEX_LABEL_ENTRZ_99;
    // HERE: input != 73, therefore we can check against A-Z
    if( input >= 65 && input < 91 ) goto QUEX_LABEL_ENTRY_94;
    // no trigger triggered 
    goto QUEX_LABEL__TERMINAL_21;
\end{lstlisting}

However, if one does too many single checks one would loose the advantage of
binary bracketting and the result would be much less efficient. The following
table shows an estimated cost in terms of comparisons with respect to the two
approaches:

\begin{center}
\begin{tabular}{l|c|c}
                         & \multicolumn{2}{c}{\bf comparisons} \\
                         & automatic & hand-written \\ \hline
    mismatch (before)    & 1 &  3 \\
    mismatch (after)     & 2 &  4 \\
    mismatch (inside)    & 4 &  4 \\	
    $[$A-H$]$            & 5 &  4 \\
    $[$J-Z$]$            & 4 &  4 \\
    ':'                  & 4 &  1 \\
    I                    & 5 &  2 \\  \hline \hline
\end{tabular}
\end{center}

A meaningful comparison of the efficiency of the two approaches, one would have
to multiply the results with the probabilities that these characters sets
occur. In the above example, the code size of the handwritten transition is
much less than the solution created automatically. Determining the optimal
algorithm that takes a parameters 'memory-size-speed-tradeoff' and computes the
optimal state transition would be an interesting contribution to the \quex
project. The author of this text welcomes any propositions to improve the
generation of transition code.  

