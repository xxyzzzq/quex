The previous sections explained the definitions of patterns, token-ids, and
mode characteristics. In this section the core of the lexical analysis is
discussed: pattern-action pairs. Lexical analysis has as a result a stream of
tokens that tell about atomic information chunks in the input stream. The most
simple pattern-action pair consists of sending a specific token whenever a
particular pattern occurs. Consider for example figure
\ref{fig:eof-pattern-action-pairs}. It shows the definition of a mode that can
handle the end of a file. The only pattern of this mode is $<<${\tt EOF}$>>$.  The action
to be executed if this pattern occurs is embraced by curly brackets.
Inside the brackets a special token is sent, a token with the id:

\begin{verbatim}
          quex::token::ID_TERMINATION
\end{verbatim}

telling that lexical analysis is over. Note, that in {\quex} one uses the
keyword {\tt self} and not a {\tt this}-pointer to refer to the lexical
analyser. Recall that {\quex} is a code generator, not a compiler.
Internally, the code fragments that are specified as pattern-action pairs or
event handlers may be pasted by {\quex} into objects other than the lexical
analyser (e.g. the lexical-analyser-mode-information structures). Also the
user might derive his own class from the lexical analyser class, so there
would be a type-issue. To unify the access to the lexical analyser, all places
were code is pasted allow to use {\tt self} as a variable that represents the
lexical analyser.  The rule here is: say '{\tt self.}' not '{\tt this->}'.

\begin{figure}
\begin{lstlisting}
mode END_OF_FILE :
<inheritable: only>  
{
    <<EOF>> {
        self.send(quex::token::ID_TERMINATION);
        RETURN;
    }
}
\end{lstlisting}
\caption{The {\tt END\_OF\_FILE} mode handling the end-of-file pattern.}
\label{fig:eof-pattern-action-pairs}
\end{figure}


A final {\tt RETURN} statement in the pattern-action pair says that the
flow of control can be turned back to the caller of function {\tt
  get\_token()}. Note again, that {\tt return} is not used. The {\tt RETURN}
macro ensures that the token queue is in a sound state before returning from
the lexical analysis. The rule here is: say '{\tt RETURN}' not '{\tt return}'.

The pattern-action pairs of the {\tt PROGRAM} mode displayed in figure
\ref{fig:program-mode} are a little more diverse. Note also, that for simple
patterns the operator '{\tt =}$>$' is used to express the sending of a token
of a specific id. The whitespace pattern does not cause any token to be sent
and one does not return to the user, thus it is followed by an empty pair of
curly brackets. The pattern {\tt P\_IDENTIFIER} creates a token that contains
the string that was matched.  {\tt P\_NUMBER} also stores a number that
corresponds the lexeme that was matched.  The pattern {\tt
  P\_STRING\_DELIMITER} does not send a token, but initiates a mode transition
to the mode {\tt STRING\_READER}.


\begin{figure}
\begin{lstlisting}
mode PROGRAM :
     END_OF_FILE
<exit:  STRING_READER>
<entry: STRING_READER>
{
    "{"         => TKN_CURLY_BRACKET_O;
    "}"         => TKN_CURLY_BRACKET_C;
    "="         => TKN_OP_ASSIGNMENT;
    "struct"    => TKN_STRUCT;
    "if"        => TKN_IF;
    ";"         => TKN_SEMICOLON;
    {P_NUMBER}  => TKN_NUMBER(atoi(yytext));
    {P_IDENTIFIER} { self.send(TKN_IDENTIFIER, yytext); RETURN; }

    {P_WHITESPACE} {
    }
    {P_STRING_DELIMITER} {
        self << STRING_READER;
    }
}
\end{lstlisting}
\caption{The {\tt PROGRAM} mode.}
\label{fig:program-mode}
\end{figure}


The {\tt STRING\_READER} mode's pattern-action pairs are shown in figure
\ref{fig:string-reader-mode}. This mode makes use of a special member of the
lexical analyser object: {\it the accumulator}. This object allows to
accumulate text from incoming patterns and flush them when desired. When this
mode is entered, the accumulator is emptied (see {\tt on\_entry}). When this
mode is left, then the accumulator is flushed (see {\tt on\_exit}). This
means, that if the accumulated text is not empty, it is send as a token with a
user defined token-id. The {\tt STRING\_READER} mode sends the accumulated
text with a token-id {\tt TKN\_STRING}.

The arrival of a string delimiter causes a mode transition back to mode {\tt
  PROGRAM}. A backslashed string delimiter allows to have a string delimter
inside the the string, so it does not cause a return to {\tt PROGRAM} mode.
instead, it adds a quote to the accumulator. The default action '.' specifies
that any other incoming text is simply accumulated. Even in this simple
example it becomes clear that the specification of entry and exit event
handlers facilitates the specification of mode transitions. It can be said
for sure, that wherever in the code a command triggers an exit to another
mode, the accumulator is flushed.

\begin{figure}
\begin{lstlisting}
mode STRING_READER :
     END_OF_FILE
<exit:  PROGRAM>
<entry: PROGRAM>
{
    on_entry {
        self.accumulator.clear();
    }

    on_exit {
        self.accumulator.flush(TKN_STRING);
    }

    {P_BACKSLASHED_STRING_DELIMITER} {
        self.accumulator.add('\"');
    }

    {P_STRING_DELIMITER} {
        /*.........................................*/ self << PROGRAM;
        RETURN;
    }

    . {
        self.accumulator.add(Lexeme);
    }
}
\end{lstlisting}
\caption{The {\tt STRING\_READER} mode.}
\label{fig:string-reader-mode}
\end{figure}

This section now finished the specification of file {\tt simple.qx} as 
an input file for the {\quex} lexical analyser generator. The next section
tells how to call {\quex} to generate a full fledged engine that does the
lexical analysis as required.


%%% Local Variables: 
%%% mode: latex
%%% TeX-master: t
%%% End: 
