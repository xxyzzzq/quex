In the simple example, only very basic patterns have to be defined. There is a
pattern for {\it whitespace}, consisting of space, tabulator and newline.
There are two bracket types for opening and closing brackets, an assignment
operator \verb|=|, the two keywords {\tt if} and {\tt struct}, identifiers
consisting of letters from A-Z in capital and lower-case, floating point
numbers and string delimeters. The listing in figure \ref{fig:pattern-code}
shows the content of the {\tt define} section in {\tt simple.qx} describing all required
patterns in terms of regular expressions.


\begin{figure}
\begin{lstlisting}
define {
    // patterns of interest for all modes
    P_WHITESPACE          [ \t\n]
    P_IDENTIFIER          [_a-zA-Z]+
    P_NUMBER              [0-9]+
    // string mode related patterns
    P_STRING_DELIMITER                "\""
    P_BACKSLASHED_STRING_DELIMITER    "\\\""
    P_BACKSLASHED_BACKSLASH           "\\\\"
}
\end{lstlisting}
\caption{Definition of pattern shorthands in a {\tt define}-section.}
\end{figure}

The filename including the extension is, of course, totally freely chooseable.
Note, that all pattern names are 'global'. They are not local to a specific
mode. That means, that all modes can access the pattern definitions. Note also,
   that inside the {\tt define}-range only '{\tt //}'-comments are allowed.
The {\tt P\_}-prefix was chosen to indicate to the reader, that this is a 
pattern definition. There is no danger of interfering with the program namespace.
These shorthand names are only considered for the definition of regular expressions.
The later generated engine does not contain them as variables.
